\section{Экспериментальная часть}
В работе использовались следующие реактивы и растворители:
\begin{table}[]
    \centering
    \caption{}
    \begin{small}
        \begin{tabular}{cccc}
            \toprule
            \textbf{Название}     & \textbf{Производитель} & \textbf{Чистота} & \textbf{Примечание} \\
            \midrule
            dcc                   & Alfa Aesar             & 99\%             &                     \\
            4-Гидроксипиперидин   & Alfa Aesar             & 97\%             &                     \\
            dmap                  &                        &                  &                     \\
            diad                  &                        &                  &                     \\
            Морфолин              & Реахим                 & Ч                &                     \\
            Пентафторацетофенон   & P\&M Invest            & 99\%             &                     \\
            Пентафторбензальдегид & ОХП НИОХ СО РАН        & 99\%             &                     \\
            Пиперазин             & Aldrich                & 99\%             &                     \\
            Трифенилфосфин        & Lancaster              & 99\%             &                     \\
            Триэтиламин           & AppliChem              & 99.5\%           &                     \\
            Фенилгидразин         & Acros Organics         & 97\%             &                     \\
            Хлористый бензоил     & Реахим                 & Ч                & Перегнан            \\
            \midrule
            Ацетон                & ЭКОС 1                 & ЧДА              &                     \\
            Ацетонитрил           & Реахим                 & ЧДА              & Над P2O5 потом сита \\
            Бензол                & Реахим                 & ЧДА              & Над Na              \\
            Гексан                & Реахим                 & ЧДА              &                     \\
            Диэтиловый эфир       & Кузбассоргхим          & ЧДА              &                     \\
            ДМФА                  & Реахим                 & ЧДА              & Над ситами          \\
            ТГФ                   & Реахим                 & ЧДА              & Над Na              \\
            Толуол                & Реахим                 & ЧДА              & Над Na              \\
            Хлористый метилен     & Реахим                 & ЧДА              &                     \\
            Этанол                & Реахим                 & ЧДА              &                     \\
            \bottomrule
        \end{tabular}
    \end{small}
\end{table}

Спектральные данные получены в Исследовательском химическом центре коллективного пользования СО~РАН.
Спектры ЯМР регистрировали на спектрометрах Bruker~AV-300~(\ce{^1H}, \SI{300.13}{\mega\hertz}; \ce{^19F},~\SI{282.37}{\mega\hertz}) и Bruker~AV-400~(\ce{^1H},~\SI{400.13}{\mega\hertz}) в дейтерохлороформе, \mbox{ДМСО-\ce{d6}} и \mbox{ацетоне-\ce{d6}}.
Значения химических сдвигов протонов приведены относительно сигналов остаточных протонов растворителей̆.
При регистрации спектров ЯМР~\ce{^19F} в качестве внутреннего стандарта использовали~\ce{C6F6}~(\chemdelta{}\textsubscript{F} = \SI{0}{\ppm}).
Спектры ЯМР~\ce{^13C} регистрировали в режиме широкополосной развязки~(broadband decoupling, BB).
Электронные спектры поглощения регистрировали на спектрофотометре Hewlett~Packard~8453.
% , спектры флуоресценции --- на спектрофлуориметре Cary~Eclipse~(Varian). 
Масс-спектры высокого разрешения получены на приборе DFS~(Thermo~Fisher~Scientifiс) в режиме прямого ввода, энергия ионизации \SI{70}{\electronvolt}.
Масс-спектры методом  MALDI-TOF получены на приборе Autoflex Speed MALDI-TOF «Bruker Daltonic» (Германия) в режиме положительного отраженного иона, частота лазера~–-- \SI{1000}{\hertz}, ускоряющее напряжение~–-- \SI{19}{\kilo\volt}\todo{уточнить}\footnote{Исследование выполнено в центре масс-спектрометрического анализа ИХБФМ СО РАН}.



% \begin{table}[]
%     \centering
%     \caption{}
%     \begin{small}
%         \begin{tabular}{cccc}
%             \toprule
%             \textbf{Название} & \textbf{Производитель} & \textbf{Чистота} & \textbf{Примечание} \\
%             \midrule
%             Ацетон            & ЭКОС 1                 & ЧДА              &                     \\
%             Ацетонитрил       & Реахим                 & ЧДА              & Над P2O5 потом сита \\
%             Бензол            & Реахим                 & ЧДА              & Над Na              \\
%             Гексан            & Реахим                 & ЧДА              &                     \\
%             Диэтиловый эфир   & Кузбассоргхим          & ЧДА              &                     \\
%             ДМФА              & Реахим                 & ЧДА              & Над ситами          \\
%             ТГФ               & Реахим                 & ЧДА              & Над Na              \\
%             Толуол            & Реахим                 & ЧДА              & Над Na              \\
%             Хлористый метилен & Реахим                 & ЧДА              &                     \\
%             Этанол            & Реахим                 & ЧДА              &                     \\
%             \bottomrule
%         \end{tabular}
%     \end{small}
% \end{table}

Альдегид~\cmpd{decafluoropyrazoline_CHO} синтезировали по~\cite{2016a,2010}.

\textbf{\iupac{4-{3,5-Бис[2,3,5,6-тетрафтор-4-(4-гидроксипиперидин-1-ил)фенил]-4,5-дигидро-1\H-пиразол-1-ил}бензальдегид}~(\cmpd{decafluoropyrazoline_substituted.piperidine}).}\todo{R19}
Раствор \SI{3.00}{\gram}~(\SI{5.9}{\mmol}) альдегида~\cmpd{decafluoropyrazoline_CHO} и \SI{1.80}{\gram}~(\SI{17.8}{\mmol}) 4-гидроксипиперидина в \SI{50}{\milli\litre} сухого \ac{dmf} нагревали до \SI{100}{\celsius}, выдерживали при этой температуре 6 часов и оставляли на ночь.
Реакционную смесь выливали в \SI{400}{\milli\litre} воды со льдом, перемешивали до таяния льда и отфильтровывали осадок.
Осадок на фильтре промывали водой до нейтральной реакции, затем гексаном и сушили на воздухе.
Желто-оранжевый порошок, выход \SI{3.70}{\gram}~(\SI{93}{\percent}).
Продукт очищали колоночной хроматографией на \ce{SiO2}, элюент~--- \ce{CH2Cl2}\,:\,ацетонитрил, градиент 5:1~-- 2:3. Собирали желтые фракции, анализировали ТСХ (\ce{CH2Cl2}\,:\,ацетонитрил, 2:1, R\textsubscript{f} $\approx$ 0.25 -- 0.3).
\begin{experimental}[]
    \data*{Т\textsubscript*{пл.}} 155--\SI{159}{\celsius}.
    \data{МС}[DFS] Найдено \ce{[M+]}: \num{668.2023}. \ce{C32H28O3N4F8}. Рассчитано: \ce{M} \num{668.2028}.
    \data{ЯМР~\ce{^1H}}[ацетон-\ce{d6}] \chemdelta, \si{\ppm}: 9.77~(с,~\ce{1H},~\ce{CHO}), 7.76~(д,~\ce{2H},~\ce{2H_{Ar}},~\textit{J}\,=\,8.8~\si{\hertz}), 7.17~(д,~\ce{2H},~\ce{2H_{Ar}},~\textit{J}\,=\,8.8~\si{\hertz}), 5.98,~4.16,~3.90~(все дд, все по~\ce{1H}, система~\emph{ABX} пиразолина,~\textit{J}\,=\,18.2, 13.1, 5.3~\si{\hertz}), 3.86\,--\,3.69~(м,~\ce{3H}), 3.63\,--\,3.47~(м,~\ce{2H}), 3.47\,--\,3.31~(м,~\ce{2H}), 3.28\,--\,3.17~(м,~\ce{2H}), 3.17\,--\,3.03~(м,~\ce{2H}), 2.00\,--\,1.82~(м,~\ce{4H}), 1.77\,--\,1.50~(м,~\ce{4H}).
    \data{ЯМР~\ce{^19F}}[ацетон-\ce{d6}] \chemdelta, м.д.: 22.27~(дд,~\ce{2F},~\textit{J}\,=\,18.4, 6.9~\si{\hertz}), 18.05~(уш.~c.,~\ce{2F}), 12.62~(дд,~\ce{2F},~\textit{J}\,=\,20.8, 6.5~\si{\hertz}), 11.67~(дд,~\ce{2F},~\textit{J}\,=\,17.6, 6.0~\si{\hertz}).
    \data{ЯМР~\ce{^13C}}[\ce{CDCl3}] \chemdelta, м.д.: 190.41, 147.40, 146.52, 146.02, 144.49, 144.05, 142.98, 141.03, 140.61, 131.67, 131.03, 130.93, 128.59, 112.66, 110.35, 110.22, 110.10, 104.00, 77.15, 76.90, 76.64, 67.18, 67.15, 51.57, 48.54, 48.51, 48.48, 48.44, 43.57, 34.85, 34.83.
\end{experimental}


\textbf{{[1-(4-Формилфенил)-4,5-дигидро-1H-пиразол-3,5-диил]бис(2,3,5,6-тетрафтор-4,1-фенилен)}бис(пиперидин-1,4-диил)дибензоат~(\cmpd{decafluoropyrazoline_benzoyl.piperidine}).}\todo{R20} \textbf{Способ 1} Суспензию \SI{0.50}{\gram}~(\SI{0.75}{\mmol}) альдегида~\textbf{\cmpd{decafluoropyrazoline_substituted.piperidine}} в \SI{10}{\milli\litre} сухого бензола доводили до кипения и прибавляли к ней \SI{0.62}{\milli\litre}~(\SI{4.5}{\mmol}) триэтиламина и \SI{0.35}{\milli\litre}~(\SI{3.0}{\mmol}) хлористого бензоила. После двух часов кипячения прибавляли еще столько же триэтиламина и хлористого бензоила и кипятили еще сутки. Реакционную смесь выливали в \SI{100}{\milli\litre} воды и добавляли бензол до разделения фаз. Органическую фазу отделяли, сушили над \ce{Na2SO4} и удаляли растворитель в вакууме. Твердый остаток очищали колоночной хроматографией на \ce{SiO2}, элюент~--- бензол\,:\,\ce{CHCl3}, градиент 1:0\,--\,0:1. Собирали желтые фракции, элюент удаляли в вакууме и повторно очищали колоночной хроматографией на \ce{SiO2}, элюент~--- смесь бензол\,:\,\ce{CH2Cl2}~1:1. Собирали желтые фракции, растворитель удаляли в вакууме. Желтое масло, выход \SI{0.49}{\gram}~(\SI{74}{\percent}).

\textbf{Способ 2} К суспензии \SI{0.20}{\gram}~(\SI{0.3}{\mmol}) альдегида~\textbf{\cmpd{decafluoropyrazoline_substituted.piperidine}} в \SI{5}{\milli\litre} сухого бензола, прибавляли~\SI{0.11}{\milli\litre}~(\SI{0.75}{\mmol}) хлористого бензоила, \SI{0.13}{\milli\litre}~(\SI{0.75}{\mmol}) триэтиламина и \SI{2}{\milli\gram}~\ac{dmap}.
Реакционную смесь кипятили 6 часов, оставляли на ночь и удаляли растворитель в вакууме.
Полученное масло очищали колоночной хроматографией на \ce{SiO2}, элюент~--- смесь ацетонитрил\,:\,\ce{CH2Cl2}, градиент 1:1 -- 8:1, собирали желтую фракцию, элюент удаляли в вакууме, полученное масло промывали смесью гексана с диэтиловым эфиром 1:1. Светло-желтый порошок, выход~\SI{0.19}{\gram}~(\SI{74}{\percent}).
\begin{experimental}
    \data*{Т\textsubscript*{пл.}} 180--\SI{183}{\celsius}.
    \data{МС}[DFS] Найдено \ce{[M+]}: \num{876.2548}. \ce{C46H36O5N4F8}. Рассчитано: \ce{M} \num{876.2553}.
    \data{ЯМР~\ce{^1H}}[\ce{CDCl3}]~\chemdelta,~м.д.: 9.77~(с,~\ce{1H},~\ce{CHO}), 8.00 -- 8.14 ~(м,~\ce{4H_{Ar}}), 7.73~(д,~\ce{2H_{Ar}},~\textit{J}\,=\,8.4~\si{\hertz}), 7.61\,--\,7.52~(м,~\ce{2H_{Ar}}), 7.50\,--\,7.39~(м,~\ce{4H_{Ar}}), 7.13~(д,~\ce{2H_Ar},,~\textit{J}\,=\,8.4~\si{\hertz}), 5.75, 3.95~(оба дд,~оба по \ce{1H}, пиразолин,~\textit{J}\,=\,17.8, 13.0, 5.9~\si{\hertz}), 5.35 -- 5.11~(м,~\ce{3H}, \ce{2CH-OH}, пиразолин), 3.65\,--\,3.41~(м,~\ce{4H}), 3.41\,--\,3.13~(м,~\ce{4H}), 2.26\,--\,2.03~(м,~\ce{4H}), 2.03\,--\,1.87~(м,~\ce{4H}).
    \data{ЯМР~\ce{^19F}}[CDCl3]~\chemdelta,~\si{\ppm}: 21.14~(д,~\ce{2F},~\textit{J}\,=\,12.2~\si{\hertz}), 16.72~(уш. с,~\ce{2F}), 11.74~(с,~\ce{2F}), 11.14\,--\,9.71~(м,~\ce{2F}).
\end{experimental}

% \textbf{4-(3,5-бис(2,3,5,6-тетрафтор-4-(пиперазин-1-ил)фенил)-4,5-дигидро-1H-пиразол-1-ил)бензальдегид~(\cmpd{decafluoropyrazoline_substituted.piperazine}).}\todo{R21} В \SI{10}{\milli\litre} сухого \ac{dmf} растворили \SI{0.50}{\gram}~(\SI{1}{\mmol}) соединения~\cmpd{decafluoropyrazoline_CHO} и \SI{0.27}{\gram}~(\SI{3}{\mmol}) пиперазина. 
% Реакционную смесь выдерживали при \SI{100}{\celsius} в течение 4 часов.
% После охлаждения выливали в \SI{100}{\milli\litre} воды, осадок отфильтровывали и сушили на воздухе. \todo{Включать?}

\textbf{\iupac{\E-\{[1-(4-\{2-[3-(Дицианометилен)-5,5-диметилциклогекс-1-ен-1-ил]винил\}фенил)-4,5-дигидро-1\H-пиразол-3,5-диил]бис(2,3,5,6-тетрафтор-4,1-фенилен)\}бис(пиперидин-1,4-диил)дибензоат}~(\cmpd{decafluoropyrazoline_piperidine_DCIF.benzoyl}).}\todo{R22 с двумя бензоилами} \textbf{Способ 1.} К суспензии \SI{0.48}{\gram}~(\SI{0.55}{\mmol}) альдегида~\textbf{\cmpd{decafluoropyrazoline_benzoyl.piperidine}} в \SI{15}{\milli\litre} бутанола прибавляли \SI{0.10}{\gram} дицианоизофорона и 5 капель морфолина. Смесь кипятили в атмосфере аргона 7 часов, растворитель удаляли в вакууме. Твердый остаток очищали колоночной хроматографией на \ce{SiO2}, элюент~--- \ce{CH2Cl2}\,:\,гексан, градиент 1:1\,--\,0:1, затем ацетонитрил. Собирали красные фракции. \todo{получили пиразол}

\textbf{Способ 2}. К суспензии \SI{0.10}{\gram}~(\SI{0.12}{\mmol}) соединения~\textbf{\cmpd{decafluoropyrazoline_DCIF.piperidine}} в \SI{5}{\milli\litre} сухого бензола прибавляли \SI{0.35}{\milli\litre}~(\SI{0.30}{\mmol}) хлористого бензоила, \SI{0.42}{\milli\litre}~(\SI{0.30}{\mmol}) триэтиламина и \SI{7}{\milli\gram} \ac{dmap}.
Реакционную смесь кипятили в атмосфере аргона 10 часов, добавив еще столько же хлористого бензоила.
Растворитель удаляли в вакууме. Очищали колоночной хроматографией на \ce{SiO2}, элюент~--- смесь ацетонитрил\,:\,\ce{CH2Cl2}, градиент 1:10 -- 1:1.
Собирали оранжевые фракции, растворитель удаляли в вакууме. Темно-оранжевый порошок, \todo{выход}
\begin{experimental}
    \data*{Т\textsubscript*{пл.}} 145--\SI{147}{\celsius}.
    \data{ЭСП}[ацетон] \chemlambda\textsubscript{max}~($\lg \varepsilon$): \SI{490}{\nano\metre}~(4.73).
    \data{МС}[MALDI-TOF] Найдено \ce{[M + H]+}: \num{1045.3609}. \ce{C56H48O4N6F8}. Рассчитано: \ce{[M + H]} \num{1045.3682}.
    \data{ЯМР~\ce{^1H}}[\ce{CDCl3}]~\chemdelta,~\si{\ppm}: 8.12\,=\,7.99~(м,~\ce{4H_{Ph}}), 7.61\,=\,7.51~(м,~\ce{2H_{Ar}}), 7.49\,=\,7.30~(м,~\ce{5H}, \ce{2H_{Ar}}, \ce{3H_{Ph}}), 7.21\,=\,7.03~(м,~\ce{3H_{Ph}}), 6.97~(д,~\ce{1H}, \ce{CH=},~\textit{J}\,=\,16.0~\si{\hertz}), 6.81~(д,~\ce{1H}, \ce{=CH},~\textit{J}\,=\,16.0~\si{\hertz}), 6.74~(с,~\ce{1H}, \ce{CH_{isoph}}), 5.72~(дд,~\ce{1H_{pyr}},~\textit{J}\,=\,12.4, 5.8~\si{\hertz}), 5.30\,=\,5.10~(м,~\ce{2H}, \ce{CH_{pip}}), 4.04\,=\,3.80~(м,~\ce{1H_{pyr}}), 3.61\,=\,3.16~(м,~\ce{9H}, \ce{4CH2_{pip}}, \ce{1H_{pyr}}), 2.56~(с,~\ce{2H}, \ce{CH2_{isoph}}), 2.42~(с,~\ce{2H}, \ce{CH2_{isoph}}), 2.20\,=\,1.86~(м,~\ce{8H}, \ce{4CH2_{pip}}), 1.24~(с,~\ce{6H}, \ce{2CH3_{isoph}}).
    \data{ЯМР~\ce{^19F}}[\ce{CDCl3}]~\chemdelta,~\si{\ppm}: 20.92\,=\,20.76~(м,~\ce{2F}), 16.76~(уш. с,~\ce{2F}), 11.60~(с,~\ce{2F}), 10.27\,=\,10.10~(м,~\ce{2F}).
\end{experimental}

\textbf{\iupac{\E-2-[3-(4-\{3,5-Бис[2,3,5,6-тетрафтор-4-(4-гидроксипиперидин-1-ил)фенил]-4,5-дигидро-1\H-пиразол-1-ил\}стирил)-5,5-диметилциклогекс-2-ен-1-илиден]малононитрил}~(\cmpd{decafluoropyrazoline_DCIF.piperidine}).}\todo{R23} К раствору \SI{0.25}{\gram}~(\SI{0.37}{\mmol}) альдегида~\textbf{\cmpd{decafluoroaldehyde.piperidine}} и \SI{0.070}{\gram}~(\SI{0.37}{\mmol}) дицианоизофорона в \SI{5}{\milli\litre} бутанола прибавляли 5 капель морфолина, кипятили в атмосфере аргона 7 часов и оставляли на ночь. Выпавший осадок отфильтровывали, промывали этанолом и диэтиловым эфиром. Темно-красный порошок, выход \SI{0.13}{\gram}~(\SI{42}{\percent}).
\begin{experimental}
    \data{ЯМР~\ce{^1H}}[\ce{CDCl3}]~\chemdelta,~\si{\ppm}: 7.37~(д,~\ce{2H_{Ar}},~\textit{J}\,=\,8.7~\si{\hertz}), 7.05~(д,~\ce{2H_{Ar}},~\textit{J}\,=\,8.7~\si{\hertz}), 6.96~(д,~\ce{1H}, \ce{CH=},~\textit{J}\,=\,15.9~\si{\hertz}), 6.80~(д,~\ce{1H}, \ce{=CH},~\textit{J}\,=\,15.9~\si{\hertz}), 6.73~(с,~\ce{1H}, \ce{CH_{isoph}}), 5.71~(дд,~\ce{1H_{pyr}},~\textit{J}\,=\,13.0, 5.6~\si{\hertz}), 3.99\,--\,3.76~(м,~\ce{3H}, \ce{2CH_{pip}}, \ce{1H_{pyr}}), 3.75\,--\,3.60~(м,~\ce{2H}, \ce{2OH_{pip}}), 3.55\,--\,3.44~(м,~\ce{2H}, \ce{CH2_{pip}}), 3.44\,--\,3.31~(м,~\ce{3H}, \ce{CH2_{pip}}, \ce{1H_{pyr}}), 3.25\,--\,3.02~(м,~\ce{4H}, \ce{2CH2_{pip}}), 2.55~(с,~\ce{2H}, \ce{CH2_{isoph}}), 2.41~(с,~\ce{2H}, \ce{CH2_{isoph}}), 1.97~(м,~\ce{4H}, \ce{2CH2_{pip}}), 1.78\,--\,1.63~(м,~\ce{4H}, \ce{2CH2_{pip}}), 1.04~(с,~\ce{6H}, \ce{2CH3_{isoph}}).
    \data{ЯМР~\ce{^19F}}[\ce{CDCl3}]~\chemdelta,~\si{\ppm}: 20.65~(дд,~\ce{2F},~\textit{J}\,=\,19.6, 7.9~\si{\hertz}), 16.69~(уш. c,~\ce{2F}), 11.44~(c,~\ce{2F}), 10.11~(д,~\ce{2F},~\textit{J}\,=\,19.6, 8.3~\si{\hertz}).
\end{experimental}


\textbf{Диэфиры~\cmpd{decafluoropyrazoline_piperidine_DCIF.{TAFS, MATBS}}~(общая методика)}
К раствору соединения \cmpd{decafluoropyrazoline_DCIF.piperidine} в \SI{6}{\milli\litre} сухого бензола добавляли соответствующий хлорангидрид, триэтиламин и \SI{1}{\milli\gram} \ac{dmap}.
Полученную смесь кипятили до окончания реакции.
Растворитель удаляли в вакууме, твердый остаток очищали колоночной хроматографией на \ce{SiO2}.
Элюент удаляли в вакууме, твердый продукт промывали гексаном или смесью гексан-эфир.

\textbf{\iupac{\E-\{[1-(4-\{2-[3-(Дицианометилен)-5,5-диметилциклогекс-1-ен-1-ил]винил\}фенил)-4,5-дигидро-1\H-пиразол-3,5-диил]бис(2,3,5,6-тетрафтор-4,1-фенилен)\}бис(пиперидин-1,4-диил) бис[4-метил-3,5-бис(\{[2,3,5,6-тетрафтор-4-(трифторметил)фенил]тио\}метил)бензоат]}~(\cmpd{decafluoropyrazoline_piperidine_DCIF.TAFS}).} \todo{пиперидин с двумя TAFS R30}
По общей методике из \SI{0.10}{\gram}~(\SI{0.12}{\milli\mole}) соединения~\cmpd{decafluoropyrazoline_DCIF.piperidine}, \SI{0.24}{\gram}~(\SI{0.36}{\milli\mole}) \ce{TAFS-Cl}, \SI{0.10}{\milli\litre}~(\SI{0.72}{\milli\mole}, 6~экв.) триэтиламина и \SI{0.001}{\gram}~\ac{dmap}.
Время реакции 2 часа. Твердый остаток после удаления растворителя очищали колоночной хроматографией на \ce{SiO2}, элюент~--- \ce{CH2Cl2}.
Темно-красный порошок, выход \SI{0.075}{\gram}~(\SI{30}{\percent}).
\begin{experimental}
    \data*{Т\textsubscript*{пл.}} 93--\SI{95}{\celsius}.
    \data{ЭСП}[ацетон] \chemlambda\textsubscript{max}~($\lg \varepsilon$): \SI{491}{\nano\metre}~(4.73).
    \data{МС}[MALDI-TOF] Найдено \ce{[M - H]-}: \num{2119.2502}. \ce{C92H56O4N6F36S4}. Рассчитано: \ce{[M - H]} \num{2119.2598}.
    \data{ЯМР~\ce{^1H}}[\ce{CD3CN}]~\chemdelta,~\si{\ppm}: 7.68\,--\,7.61~(м,~\ce{4H_{TAFS}}), 7.57~(д,~\ce{2H_{Ar}},~\textit{J}\,=\,8.6~\si{\hertz}), 7.22~(д,~\ce{1H}, \ce{CH=},~\textit{J}\,=\,16.2~\si{\hertz}), 7.15~(д,~\ce{2H_{Ar}},~\textit{J}\,=\,8.6~\si{\hertz}), 7.09~(д,~\ce{1H}, \ce{=CH},~\textit{J}\,=\,16.2~\si{\hertz}), 6.87~(с,~\ce{1H}, \ce{CH_{isoph}}), 5.95~(дд,~\ce{1H_pyr},~\textit{J}\,=\,11.3, 3.4~\si{\hertz}), 5.28\,--\,5.09~(м,~\ce{2H}, \ce{2CH_{pip}}), 4.49\,--\,4.37~(м,~\ce{8H}, \ce{4CH2_{TAFS}}), 4.24\,--\,3.99~(м,~\ce{1H_{pyr}}), 3.73\,--\,3.23~(м,~\ce{9H}, \ce{1H_{pyr}}, \ce{4CH2_{pip}}), 2.60~(с,~\ce{2H}, \ce{CH2_{isoph}}), 2.21~(с,~\ce{6H}, \ce{2CH3_{TAFS}}), 2.18\,--\,2.11~(м,~\ce{4H}, \ce{2CH2_{pip}}), 2.00\,--\,1.80~(м,~\ce{4H}, \ce{2CH2_{pip}}), 1.45\,--\,1.34~(м,~\ce{2H}, \ce{CH2_{isoph}}), 1.14~(с,~\ce{6H}, \ce{2CH3_{isoph}}).
    \data{ЯМР~\ce{^19F}}[\ce{CD3CN}]~\chemdelta,~\si{\ppm}: 109.21\,--\,105.32~(м,~\ce{12F}, \ce{4CF3}), 32.41\,--\,31.58~(м,~\ce{8F}), 22.10\,--\,21.54~(м,~\ce{10F}), 17.84~(уш. с,~\ce{2F}), 12.98\,--\,12.06~(м,~\ce{2F}), 11.83\,--\,10.84~(м,~\ce{2F}).
\end{experimental}

\textbf{\iupac{\E-\{[1-(4-\{2-[3-(Дицианометилен)-5,5-диметилциклогекс-1-ен-1-ил]винил\}фенил)-4,5-дигидро-1\H-пиразол-3,5-диил]бис(2,3,5,6-тетрафтор-4,1-фенилен)\}бис(пиперидин-1,4-диил) бис[3,5-бис(\{[4-(трет-бутил)фенил]тио\}метил)-4-метилбензоат]}~(\cmpd{decafluoropyrazoline_piperidine_DCIF.TATBS}).}\todo{пиперидин с двумя TATBS R33}
По общей методике из \SI{0.06}{\gram}~(\SI{0.07}{\milli\mole}) соединения~\cmpd{decafluoropyrazoline_DCIF.piperidine}, \SI{0.12}{\gram}~(\SI{0.22}{\milli\mole}) \ce{TATBS-Cl}, \SI{0.14}{\milli\litre}~(\SI{1.1}{\milli\mole}) триэтиламина и \SI{0.001}{\gram}~\ac{dmap}.
Время реакции 6 часов. Твердый остаток после удаления растворителя очищали колоночной хроматографией на \ce{SiO2}, элюент~--- бензол.
Темно-красный порошок, выход \SI{0.070}{\gram}~(\SI{55}{\percent}). \todo{ЭСП}
\begin{experimental}
    \data{МС}[MALDI-TOF] Найдено \ce{[M + H]+}: \num{1785.7141}. \ce{C104H108O4N6F8S4}. Рассчитано: \ce{[M + H]} \num{1785.7260}.
    \data{ЯМР~\ce{^1H}}[ацетон-d\textsubscript{6}]~\chemdelta,~\si{\ppm}: 7.78~(с,~\ce{2H_{MATBS}}), 7.75~(с,~\ce{2H_{MATBS}}), 7.60~(д,~\ce{2H_{AR}},~\textit{J}\,=\,8.7~\si{\hertz}), 7.37\,--\,7.25~(м,~\ce{16H_{MATBS}}), 7.20~(д,~\ce{2H_{Ar}},~\textit{J}\,=\,7.3~\si{\hertz}), 7.13\,--\,7.05~(м,~\ce{2H}, \ce{CH=CH}), 6.78~(с,~\ce{1H}, \ce{CH_{isoph}}), 5.98~(дд,~\ce{1H_{pyr}},~\textit{J}\,=\,13.0, 5.6~\si{\hertz}), 5.21\,--\,5.04~(м,~\ce{2H}, \ce{2CH_{pip}}), 4.27~(с,~\ce{4H}, \ce{2CH2_{MATBS}}), 4.25~(с,~\ce{4H}, \ce{2CH2_{MATBS}}), 4.14~(дд,~\ce{1H_{pyr}},~\textit{J}\,=\,18.0, 13.0~\si{\hertz}), 3.65\,--\,3.21~(м,~\ce{9H}, \ce{4CH2_{pip}}, \ce{1H_{pyr}}), 2.62~(с,~\ce{2H}, \ce{CH2_{isoph}}), 2.59\,--\,2.56~(м,~\ce{2H}, \ce{CH2_{isoph}}), 2.50~(с,~\ce{3H}, \ce{CH3_{MATBS}}), 2.48~(с,~\ce{3H}, \ce{CH3_{MATBS}}), 2.14\,--\,2.07~(м,~\ce{4H}, \ce{2CH2_{pip}}), 1.99\,--\,1.75~(м,~\ce{4H}, \ce{2CH2_{pip}}), 1.27~(с,~\ce{18H}, \ce{tBu_{MATBS}}), 1.25~(с,~\ce{18H}, \ce{tBu_{MATBS}}), 1.07~(с,~\ce{6H}, \ce{2CH3_{isoph}}).
    \data{ЯМР~\ce{^19F}}[ацетон-d\textsubscript{6}]~\chemdelta,~\si{\ppm}: 22.59\,--\,21.86~(м,~\ce{2F}), 18.47~(уш. с,~\ce{2F}), 13.01\,--\,12.49~(м,~\ce{2F}), 12.23\,--\,11.43~(м,~\ce{2F}).
\end{experimental}

\textbf{Моноэфиры~\cmpd{pentafluoropyrazoline_piperidine_DCIF.{benzoyl, TAFS, TATBS, MATBS}}~(общая методика)}
К раствору \SI{0.10}{\gram}~(\SI{0.15}{\milli\mole}) соединения \cmpd{pentafluoropyrazoline_DCIF.piperidine} в \SI{6}{\milli\litre} сухого бензола добавляли 0.2-0.3~\si{\milli\mole} соответствующего хлорангидрида , 0.2-0.3~\si{\milli\mole} триэтиламина и \SI{1}{\milli\gram} \ac{dmap}.
Полученную смесь кипятили до окончания реакции.
Растворитель удаляли в вакууме, твердый остаток очищали колоночной хроматографией на \ce{SiO2}, элюент~--- бензол.
Элюент удаляли в вакууме, твердый продукт промывали гексаном или смесью гексан-эфир.

\textbf{\iupac{\E-1-\{4-[1-(4-\{2-[3-(Дицианометилен)-5,5-диметилциклогекс-1-ен-1-ил]винил\}фенил)-3-фенил-4,5-дигидро-1\H-пиразол-5-ил]-2,3,5,6-тетрафторфенил\}пиперидин-4-ил бензоат}~(\cmpd{pentafluoropyrazoline_piperidine_DCIF.benzoyl}).} \todo{пиперидин с одним бензоилом}
По общей методике из \SI{0.10}{\gram}~(\SI{0.15}{\milli\mole}) соединения~\cmpd{pentafluoropyrazoline_DCIF.piperidine}, \SI{0.03}{\milli\litre}~(\SI{0.23}{\milli\mole}) хлористого бензоила, \SI{0.04}{\milli\litre}~(\SI{0.23}{\milli\mole}) триэтиламина и \SI{0.001}{\gram})~\ac{dmap}.
Время реакции 4 часа.
Темно-красный порошок, выход \SI{0.11}{\gram}~(\SI{92}{\percent}).
\begin{experimental}
    \data*{Т\textsubscript*{пл.}} 145--\SI{147}{\celsius}.
    \data{ЭСП}[ацетон] \chemlambda\textsubscript{max}~($\lg \varepsilon$): \SI{499}{\nano\metre}~(4.68).
    \data{МС}[DFS] Найдено \ce{[M+]}: \num{}. \ce{C46H39O2N5F4}. Рассчитано: \ce{M} \num{}. \todo{написать}
    \data{ЯМР~\ce{^1H}}[ацетон-d\textsubscript{6}]~\chemdelta,~\si{\ppm}: 8.03~(д,~\ce{2H_{Ar}},~\textit{J}\,=\,7.6~\si{\hertz}), 7.82~(д,~\ce{2H_{Ar}},~\textit{J}\,=\,7.6~\si{\hertz}), 7.67\,--\,7.54~(м,~\ce{3H_{Ph}}), 7.54\,--\,7.36~(м,~\ce{5H_{Ph}}), 7.22~(д,~\ce{1H}, \ce{CH=},~\textit{J}\,=\,15.3~\si{\hertz}), 7.18\,--\,7.04~(м,~\ce{3H}, \ce{=CH}, \ce{2H_{Ph}}), 6.76~(с,~\ce{1H}, \ce{CH_{isoph}}), 5.94~(дд,~\ce{1H_{pyr}},~\textit{J}\,=\,13.0, 5.6~\si{\hertz}), 5.28\,--\,5.06~(м,~\ce{1H}, \ce{CH_{pip}}), 4.08~(дд,~\ce{1H_{pyr}},~\textit{J}\,=\,18.0, 13.0~\si{\hertz}), 3.65\,--\,3.40~(м,~\ce{3H}, \ce{1H_{pyr}}, \ce{CH2_{pip}}), 3.28~(с,~\ce{2H}, \ce{CH2_{pip}}), 2.61~(с,~\ce{2H}, \ce{CH2_{isoph}}), 2.56~(с,~\ce{2H}, \ce{CH2_{isoph}}), 1.94\,--\,1.75~(м,~\ce{4H}, \ce{2CH2_{pip}}), 1.06~(с,~\ce{6H}, \ce{2CH3_{isoph}}).
    \data{ЯМР~\ce{^19F}}[ацетон-d\textsubscript{6}]~\chemdelta,~\si{\ppm}: 18.23~(уш. с,~\ce{2F}), 12.70~(д,~\ce{2F},~\textit{J}\,=\,21.2~\si{\hertz}).
\end{experimental}

\textbf{\iupac{\E-1-\{4-[1-(4-\{2-[3-(Дицианометилен)-5,5-диметилциклогекс-1-ен-1-ил]винил\}фенил)-3-фенил-4,5-дигидро-1\H-пиразол-5-ил]-2,3,5,6-тетрафторфенил\}пиперидин-4-ил 4-метил-3,5-бис(\{[2,3,5,6-тетрафтор-4-(трифторметил)фенил]тио\}метил)бензоат}~(\cmpd{pentafluoropyrazoline_piperidine_DCIF.TAFS}).} \todo{пиперидин с одним TAFS R36}
По общей методике из \SI{0.10}{\gram}~(\SI{0.15}{\milli\mole}) соединения~\cmpd{pentafluoropyrazoline_DCIF.piperidine}, \SI{0.15}{\gram}~(\SI{0.23}{\milli\mole}) \ce{TAFS-Cl}, \SI{0.04}{\milli\litre}~(\SI{0.23}{\milli\mole}) триэтиламина и \SI{0.001}{\gram}~\ac{dmap}.
Время реакции 2.5 часа.
Темно-красный порошок, выход \SI{0.19}{\gram}~(\SI{97}{\percent}).
\begin{experimental}
    \data*{Т\textsubscript*{пл.}} 108--\SI{111}{\celsius}.
    \data{ЭСП}[ацетон] \chemlambda\textsubscript{max}~($\lg \varepsilon$): \SI{499}{\nano\metre}~(4.61).
    \data{МС}[MALDI-TOF] Найдено \ce{[M + H]+}: \num{1308.2568}. \ce{C63H43O2N5F18S2}. Рассчитано: \ce{[M + H]} \num{1308.2644}.
    \data{ЯМР~\ce{^1H}}[ацетон-d\textsubscript{6}]~\chemdelta,~\si{\ppm}: 7.84~(д,~\ce{2H_{Ar}},~\textit{J}\,=\,8.8~\si{\hertz}), 7.65~(с,~\ce{2H_{TAFS}}), 7.58~(д,~\ce{2H_{Ar}},~\textit{J}\,=\,8.8~\si{\hertz}), 7.50\,--\,7.38~(м,~\ce{3H_{Ph}}), 7.34~(с,~\ce{1H_{Ph}}), 7.28\,--\,7.04~(м,~\ce{3H}, \ce{1H_{Ph}}, \ce{CH=CH}), 6.77~(с,~\ce{1H}, \ce{CH_{isoph}}), 5.95~(дд,~\ce{1H_{pyr}},~\textit{J}\,=\,12.9, 5.7~\si{\hertz}), 5.17\,--\,5.00~(м,~\ce{1H}, \ce{CH_{pip}}), 4.47~(с,~\ce{4H}, \ce{2CH2_{TAFS}}), 4.09~(дд,~\ce{1H_{pyr}},~\textit{J}\,=\,17.8, 12.9~\si{\hertz}), 3.64\,--\,3.18~(м,~\ce{5H}, \ce{1H_{pyr}}, \ce{2CH2_{pip}}), 2.66~(с,~\ce{3H}, \ce{CH3_{TAFS}}), 2.61~(с,~\ce{2H}, \ce{CH2_{isoph}}), 2.56~(с,~\ce{2H}, \ce{CH2_{isoph}}), 1.83~(с,~\ce{2H}, \ce{CH2_{pip}}), 1.30\,--\,1.22~(м,~\ce{2H}, \ce{CH2_{pip}}), 1.07~(с,~\ce{6H}, \ce{2CH3_{isoph}}).
    \data{ЯМР~\ce{^19F}}[ацетон-d\textsubscript{6}]~\chemdelta,~\si{\ppm}: 107.51~(т,~\ce{6F}, \ce{2CF3},~\textit{J}\,=\,21.9~\si{\hertz}), 32.23\,--\,31.88~(м,~\ce{4F}), 22.26\,--\,21.42~(м,~\ce{4F}), 18.23~(уш. с,~\ce{2F}), 12.65~(дд,~\ce{2F},~\textit{J}\,=\,21.9, 8.2~\si{\hertz}).
\end{experimental}


\textbf{\iupac{\E-1-\{4-[1-(4-\{2-[3-(Дицианометилен)-5,5-диметилциклогекс-1-ен-1-ил]винил\}фенил)-3-фенил-4,5-дигидро-1\H-пиразол-5-ил]-2,3,5,6-тетрафторфенил\}пиперидин-4-ил 3,5-бис(\{[4-(трет-бутил)фенил]тио\}метил)4-метилбензоат}~(\cmpd{pentafluoropyrazoline_piperidine_DCIF.TATBS}).} \todo{пиперидин с одним TATBS R31}
\textbf{Способ 1.}~По общей методике из \SI{0.10}{\gram}~(\SI{0.15}{\milli\mole}) соединения~\cmpd{pentafluoropyrazoline_DCIF.piperidine}, \SI{0.12}{\gram}~(\SI{0.23}{\milli\mole}) \ce{TATBS-Cl}, \SI{0.06}{\milli\litre}~(\SI{0.4}{\milli\mole}) триэтиламина и \SI{0.001}{\gram}~\ac{dmap}.
Время реакции 3 часа.
Темно-красный порошок, выход \SI{0.10}{\gram}~(\SI{59}{\percent}).

\textbf{Способ 2.}
К раствору \SI{0.05}{\gram}~(\SI{0.75}{\milli\mole}) соединения~\cmpd{pentafluoropyrazoline_DCIF.piperidine}, \SI{0.04}{\gram}~(\SI{0.75}{\milli\mole}) \ce{TATBS-OH} и \SI{0.03}{\gram}~(\SI{0.11}{\milli\mole}) трифенилфосфина в \SI{6}{\milli\litre} сухого ТГФ при перемешивании в атмосфере аргона по каплям прибавляли раствор \SI{0.02}{\milli\litre}~\ac{diad} в \SI{4}{\milli\litre} сухого ТГФ. Реакционную смесь выдерживали при комнатной температуре 2.5 часа, растворитель удаляли в вакууме. Твердый остаток очищали колоночной хроматографией на \ce{SiO2}, элюент~--- бензол. Собирали красные фракции, элюент удаляли в вакууме. Темно-красный порошок, выход \SI{0.06}{\gram}~(\SI{70}{\percent}).

\textbf{Способ 3.}
Раствор \SI{0.05}{\gram}~(\SI{0.75}{\milli\mole}) соединения~\cmpd{pentafluoropyrazoline_DCIF.piperidine}, \SI{0.04}{\gram}~(\SI{0.75}{\milli\mole}) \ce{TATBS-OH}, \SI{0.02}{\gram}~(\SI{0.83}{\milli\mole})~\ac{dcc} и \SI{0.001}{\gram}~\ac{dmap} в \SI{6}{\milli\litre} сухого бензола выдерживали 5 часов, после кипятили 7 часов. Растворитель удаляли в вакууме, Твердый остаток очищали колоночной хроматографией на \ce{SiO2}, элюент~--- бензол. Собирали красные фракции, элюент удаляли в вакууме. Темно-красный порошок, выход \SI{0.02}{\gram}~(\SI{22}{\percent}).
\begin{experimental}
    \data*{Т\textsubscript*{пл.}} 108--\SI{110}{\celsius}.
    \data{ЭСП}[ацетон] \chemlambda\textsubscript{max}~($\lg \varepsilon$): \SI{499}{\nano\metre}~(4.67).
    \data{МС}[MALDI-TOF] Найдено \ce{[M + H]+}: \num{1140.4823}. \ce{C69H69O2N5F4S2}. Рассчитано: \ce{[M + H]} \num{1140.4902}.
    \data{ЯМР~\ce{^1H}}[ацетон-d\textsubscript{6}]~\chemdelta,~\si{\ppm}: 7.82~(д,~\ce{2H_{Ar}},~\textit{J}\,=\,8.5~\si{\hertz}), 7.72~(с,~\ce{2H_{MATBS}}), 7.57~(д,~\ce{2H_{Ar}},~\textit{J}\,=\,8.5~\si{\hertz}), 7.48\,--\,7.37~(м,~\ce{3H_{Ph}}), 7.32~(с,~\ce{2H_{Ph}}), 7.30\,--\,7.21~(м,~\ce{8H_{MATBS}}), 7.20\,--\,7.00~(м,~\ce{2H}, \ce{CH=CH}), 6.75~(с,~\ce{1H}, \ce{CH_{isoph}}), 5.94~(дд,~\ce{1H_{pyr}},~\textit{J}\,=\,12.9, 5.6~\si{\hertz}), 5.06~(м,~\ce{1H}, \ce{CH_{pip}}), 4.23~(с,~\ce{4H}, \ce{2CH2_{MATBS}}), 4.07~(дд,~\ce{1H_{pyr}},~\textit{J}\,=\,17.8, 12.9~\si{\hertz}), 3.62\,--\,3.38~(м,~\ce{3H}, \ce{1H_{pyr}}, \ce{CH2_{pip}}), 3.34\,--\,3.16~(м,~\ce{2H}, \ce{CH2_{pip}}), 2.60~(с,~\ce{2H}, \ce{CH2_{isoph}}), 2.55~(с,~\ce{2H}, \ce{CH2_{isoph}}), 2.47~(с,~\ce{3H}, \ce{CH3_{MATBS}}), 1.90\,--\,1.73~(м,~\ce{2H}, \ce{CH2_{pip}}), 1.38\,--\,1.18~(м,~\ce{20H}, \ce{2tBu_{MATBS}}, \ce{CH2_{pip}}), 1.05~(c,~\ce{6H}, \ce{2CH3_{isoph}}).
    \data{ЯМР~\ce{^19F}}[ацетон-d\textsubscript{6}]~\chemdelta,~\si{\ppm}: 18.29~(уш. с,~\ce{2F}), 12.74~(д,~\ce{2F},~\textit{J}\,=\,20.1~\si{\hertz}).
\end{experimental}

\textbf{\iupac{\E-1-\{4-[1-(4-\{2-[3-(Дицианометилен)-5,5-диметилциклогекс-1-ен-1-ил]винил\}фенил)-3-фенил-4,5-дигидро-1\H-пиразол-5-ил]-2,3,5,6-тетрафторфенил\}пиперидин-4-ил 3,5-бис(\{[4-(трет-бутил)фенил]тио\}метил)-2,4,6-триметилбензоат}~(\cmpd{pentafluoropyrazoline_piperidine_DCIF.MATBS}).} \todo{пиперидин с одним MATBS R32}
По общей методике из \SI{0.09}{\gram}~(\SI{0.14}{\milli\mole}) соединения~\cmpd{pentafluoropyrazoline_DCIF.piperidine}, \SI{0.11}{\gram}~(\SI{0.21}{\milli\mole}) \ce{MATBS-Cl}, \SI{0.06}{\milli\litre}~(\SI{0.4}{\milli\mole}) триэтиламина и \SI{0.001}{\gram}~(0.05~экв.) \ac{dmap}.
Время реакции 12 часов.
Темно-красный порошок, выход \SI{0.012}{\gram}~(\SI{7.5}{\percent}).
\begin{experimental}
    \data*{Т\textsubscript*{пл.}} 147--\SI{150}{\celsius}.
    \data{ЭСП}[ацетон] \chemlambda\textsubscript{max}~($\lg \varepsilon$): \SI{497}{\nano\metre}~(4.64).
    \data{МС}[MALDI-TOF] Найдено \ce{[M + H]+}: \num{1168.5130}. \ce{C71H73O2N5F4S2}. Рассчитано: \ce{[M + H]} \num{1168.5215}.
    \data{ЯМР~\ce{^1H}}[\ce{CDCl3}]~\chemdelta,~\si{\ppm}: 7.74~(д,~\ce{2H_{Ar}},~\textit{J}\,=\,7.3~\si{\hertz}), 7.47\,--\,7.34~(м,~\ce{5H_{Ph}}), 7.33\,--\,7.24~(м,~\ce{8H_{MATBS}}), 7.08~(д,~\ce{2H_{Ar}},~\textit{J}\,=\,7.3~\si{\hertz}), 6.97~(д,~\ce{1H}, \ce{CH=},~\textit{J}\,=\,16.1~\si{\hertz}), 6.81~(д,~\ce{1H}, \ce{=CH},~\textit{J}\,=\,16.1~\si{\hertz}), 6.74~(с,~\ce{1H}, \ce{CH_{isoph}}), 5.74~(дд,~\ce{1H_{pyr}},~\textit{J}\,=\,13.0, 6.0~\si{\hertz}), 5.28~(с,~\ce{6H}, \ce{2CH3_{MATBS}}), 5.17\,--\,5.03~(м,~\ce{1H}, \ce{CH_{pip}}), 4.10~(с,~\ce{4H}, \ce{2CH2_{MATBS}}), 3.87~(дд,~\ce{1H_{pyr}},~\textit{J}\,=\,17.4, 13.0~\si{\hertz}), 3.47\,--\,3.33~(м,~\ce{3H}, \ce{1H_{pyr}}, \ce{CH2_{pip}}), 3.25\,--\,3.13~(м,~\ce{2H}, \ce{CH2_{pip}}), 2.56~(с,~\ce{2H}, \ce{CH2_{isoph}}), 2.47\,--\,2.36~(м,~\ce{5H}, \ce{CH3_{MATBS}}, \ce{CH2_{isoph}}), 2.10\,--\,1.96~(м,~\ce{2H}, \ce{CH2_{pip}}), 1.94\,--\,1.79~(м,~\ce{2H}, \ce{CH2_{pip}}), 1.27~(с,~\ce{18H}, \ce{2tBu_{MATBS}}), 1.04~(с,~\ce{6H}, \ce{2CH3_{isoph}}).
    \data{ЯМР~\ce{^19F}}[\ce{CDCl3}]~\chemdelta,~\si{\ppm}: 16.86~(уш. с,~\ce{2F}), 11.57~(д,~\ce{2F},~\textit{J}\,=\,20.4~\si{\hertz}).
\end{experimental}
