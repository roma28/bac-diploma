\section{Экспериментальная часть}

Спектральные данные получены в Исследовательском химическом центре коллективного пользования СО~РАН. Спектры ЯМР регистрировали на спектрометрах Bruker~AV-300 (\ce{^1H}, \SI{300.13}{\mega\hertz}; \ce{^19F}, \SI{282.37}{\mega\hertz}) и Bruker~AV-400 (\ce{^1H}, \SI{400.13}{\mega\hertz}) в дейтерохлороформе, ДМСО-\ce{d6} и ацетоне-\ce{d6}. Значения химических сдвигов протонов приведены относительно сигналов остаточных протонов растворителей. При регистрации спектров ЯМР~\ce{^19F} в качестве внутреннего стандарта использовали \ce{C6F6}. Спектры ЯМР \ce{^13C} регистрировали в режиме широкополосной развязки~(broadband decoupling, BB). Электронные спектры поглощения регистрировали на спектрофотометре Hewlett~Packard~8453, спектры флуоресценции --- на спектрофлуориметре Cary~Eclipсe~(Varian). Масс-спектры высокого разрешения получены на приборе DFS~(Thermo~Fisher~Scientifiс) в режиме прямого ввода, энергия ионизации \SI{70}{\electronvolt}.

\textbf{4-(3,5-бис(2,3,5,6-тетрафтор-4-(4-гидроксипиперидин-1-ил)фенил-4,5-дигидро-1H-пиразол-1-ил)бензальдегид~(\cmpd{decafluoroaldehyde.piperidine}).}\todo{R19} Раствор \SI{3.00}{\gram}~(\SI{5.9}{\mmol}) альдегида~\textbf{\cmpd{decafluoro}} и \SI{1.80}{\gram}~(\SI{17.8}{\mmol}) 4-гидроксипиперидина в \SI{50}{\milli\litre} сухого \ac{dmf} нагревали до \SI{100}{\celsius}, выдерживали при этой температуре 6 часов и оставляли на ночь. Реакционную смесь выливали в \SI{400}{\milli\litre} воды со льдом, перемешивали до таяния льда и отфильтровывали осадок. Осадок на фильтре промывали водой до нейтральной реакции, затем гексаном и сушили на воздухе. Желто-оранжевый порошок, выход \SI{3.70}{\gram}~(\SI{93}{\percent}). Продукт очищали колоночной хроматографией на \ce{SiO2}, элюент~--- \ce{CH2Cl2}\,:\,ацетонитрил, градиент 5:1~-- 2:3. Собирали желтые фракции, анализировали ТСХ (\ce{CH2Cl2}\,:\,ацетонитрил, 2:1, R\textsubscript{f} $\approx$ 0.25 -- 0.3).
\begin{experimental}[format=\bfseries]
    \data*{Т\textsubscript*{пл.}} 155--\SI{159}{\celsius}.

    \data{МС}[DFS] Найдено \ce{[M+]}: \num{668.2023}. \ce{C32H28O3N4F8}. Рассчитано: \ce{M} \num{668.2028}.

    \data{ЯМР~\ce{^1H}}[ацетон-\ce{d6}] $\delta$, м.д.: 9.77~(с,~\ce{1H}), 7.76~(д,~\ce{2H},~\textit{J}~=~8.8~\si{\hertz}), 7.17~(д,~\ce{2H},~\textit{J}~=~8.8~\si{\hertz}), 5.98,~4.16,~3.90~(все дд, все по~\ce{1H},~\textit{J}~=~18.2, 13.1, 5.3~\si{\hertz}), 3.86~--~3.69~(м,~\ce{3H}), 3.63~--~3.47~(м,~\ce{2H}), 3.47~--~3.31~(м,~\ce{2H}), 3.28~--~3.17~(м,~\ce{2H}), 3.17~--~3.03~(м,~\ce{2H}), 2.00~--~1.82~(м,~\ce{4H}), 1.77~--~1.50~(м,~\ce{4H}).

    \data{ЯМР~\ce{^19F}}[ацетон-\ce{d6}] $\delta$, м.д.: 22.27~(дд,~\ce{2F},~\textit{J}~=~18.4, 6.9~\si{\hertz}), 18.05~(уш.~c.,~\ce{2F}), 12.62~(дд,~\ce{2F},~\textit{J}~=~20.8, 6.5~\si{\hertz}), 11.67~(дд,~\ce{2F},~\textit{J}~=~17.6, 6.0~\si{\hertz}).

    \data{ЯМР~\ce{^13C}}[\ce{CDCl3}] $\delta$, м.д.: 190.41, 147.40, 146.52, 146.02, 144.49, 144.05, 142.98, 141.03, 140.61, 131.67, 131.03, 130.93, 128.59, 112.66, 110.35, 110.22, 110.10, 104.00, 77.15, 76.90, 76.64, 67.18, 67.15, 51.57, 48.54, 48.51, 48.48, 48.44, 43.57, 34.85, 34.83.
\end{experimental}

\textbf{((1-(4-формилфенил)-4,5-дигидро-1H-пиразол-3,5-диил)бис(2,3,5,6-тетрафтор-4,1-фенилен))бис(пиперидин-1,4-диил)дибензоат~(\cmpd{benzoate.piperidine}).}\todo{R20} Суспензию \SI{0.50}{\gram}~(\SI{0.75}{\mmol}) альдегида~\textbf{\cmpd{decafluoroaldehyde.piperidine}} в \SI{10}{\milli\litre} сухого бензола доводили до кипения и прибавляли к ней \SI{0.62}{\milli\litre}~(\SI{4.5}{\mmol}) триэтиламина и \SI{0.35}{\milli\litre}~(\SI{3.0}{\mmol}) хлористого бензоила. После двух часов кипячения прибавляли еще столько же триэтиламина и хлористого бензоила, кипятили еще сутки и выдерживали неделю \todo{Странная методика}. Реакционную смесь выливали в \SI{100}{\milli\litre} воды и добавляли бензол до разделения фаз. Органическую фазу отделяли, сушили над \ce{Na2SO4} и удаляли растворитель в вакууме. Твердый остаток очищали колончной хроматографией на \ce{SiO2}, элюент~--- бензол\,:\,\ce{CHCl3}, градиент 1:0~--~0:1. Собирали желтые фракции, элюент удаляли в вакууме и повторно очищали колончной хроматографией на \ce{SiO2}, элюент~--- смесь бензол\,:\,\ce{CH2Cl2}~1:1. Собирали желтые фракции, растворитель удаляли в вакууме. Желтое масло, выход \SI{0.49}{\gram}~(\SI{74}{\percent}). Аналитически чистый образец получали промыванием смесью гексана с диэтиловым эфиром.
\begin{experimental}[format=\bfseries]
    \data*{Т\textsubscript*{пл.}} 180--\SI{183}{\celsius}.
    \data{МС}[DFS] Найдено \ce{[M+]}: \num{876.2548}. \ce{C46H36O5N4F8}. Рассчитано: \ce{M} \num{876.2553}.
\end{experimental}

\textbf{4-(3,5-бис(2,3,5,6-тетрафтор-4-(пиперазин-1-ил)фенил)-4,5-дигидро-1H-пиразол-1-ил)бензальдегид~(\cmpd{decafluoroaldehyde.piperazine}).}\todo{R21}

\textbf{(\textit{E})-((1-(4-(2-(3-(дицианометилен)-5,5-диметилциклогекс-1-ен-1-ил)винил)фенил)-4,5-дигидро-1H-пиразол-3,5-диил)бис(2,3,5,6-тетрафтор-4,1-фенилен))бис(пиперидин-1,4-диил)дибензоат~(\cmpd{dcif.piperidinebenzoate}).}\todo{R22} \textbf{Способ 1.} К суспензии \SI{0.48}{\gram}~(\SI{0.55}{\mmol}) альдегида~\textbf{\cmpd{benzoate.piperidine}} в \SI{15}{\milli\litre} бутанола прибавляли \SI{0.10}{\gram} дицианоизофорона \textbf{\cmpd{dicianoisophorone}} и 5 капель морфолина. Смесь кипятили в атмосфере аргона 7 часов, растворитель удаляли в вакууме. Твердый остататок очищали колончной хроматографией на \ce{SiO2}, элюент~--- \ce{CH2Cl2}\,:\,гексан, градиент 1:1~--~0:1, затем ацетонитрил. Собирали красные фракции.

\textbf{Способ 2}. \todo{R24} К суспензии \SI{0.10}{\gram}~(\SI{0.12}{\mmol}) соединения~\textbf{\cmpd{piperidine_dcif}} в \SI{5}{\milli\litre} сухого бензола прибавляли \SI{0.35}{\milli\litre}~(\SI{0.30}{\mmol}) хлористого бензоила, \SI{0.42}{\milli\litre}~(\SI{0.30}{\mmol}) триэтиламина и \SI{7}{\milli\gram} \ac{dmap}. Реакционную смесь кипятили в атмосфере аргона 10 часов, добавив еще столько же хлористого бензоила. Растворитель удаляли в вакууме.

\textbf{(\textit{E})-2-(3-(4-(3,5-бис(2,3,5,6-тетрафтор-4-(4-гидроксипиперидин-1-ил)фенил)-4,5-дигидро-1H-пиразол-1-ил)стирил)-5,5-диметилциклогекс-2-ен-1-илиден)малононитрил~(\cmpd{piperidine_dcif}).}\todo{R23} К раствору \SI{0.25}{\gram}~(\SI{0.37}{\mmol}) альдегида~\textbf{\cmpd{decafluoroaldehyde.piperidine}} и \SI{0.070}{\gram}~(\SI{0.37}{\mmol}) дицианоизофорона~\textbf{\cmpd{dicianoisophorone}} в \SI{5}{\milli\litre} бутанола прибавляли 5 капель морфолина, кипятили в атмосфере аргона 7 часов и оставляли на ночь. Выпавший осадок отфильтровывали, промывали этанолом и диэтиловым эфиром. Темно-красный порошок, выход \SI{0.13}{\gram}~(\SI{42}{\percent}).

