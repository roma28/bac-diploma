\section{Экспериментальная часть}
В работе использовались следующие коммерчески доступные реактивы и растворители, которые дополнительно не очищали, если не указано иное.
\begin{table}[h!]
    \centering
    \caption{Использованные коммерчески доступные реактивы и растворители и методы их очистки}
    \begin{small}
        \begin{tabular}{cccc}
            \toprule
            \textbf{Название}                   & \textbf{Производитель} & \textbf{Чистота} & \textbf{Примечание}                 \\
            \midrule
            \ac{dcc}                            & Alfa Aesar             & 99\%             & ---                                 \\
            4-Гидроксипиперидин                 & Alfa Aesar             & 97\%             & ---                                 \\
            \ac{dmap}                           & Sigma Aldrich          & 99\%             & ---                                 \\
            \ac{diad}                           & Sigma Aldrich          & 98\%             & ---                                 \\
            Морфолин                            & Реахим                 & Ч                & ---                                 \\
            Пентафторацетофенон                 & P\&M Invest            & 99\%             & ---                                 \\
            Пентафторбензальдегид               & ОХП НИОХ СО РАН        & 99\%             & ---                                 \\
            Пиперазин                           & Aldrich                & 99\%             & ---                                 \\
            Трифенилфосфин                      & Lancaster              & 99\%             & ---                                 \\
            Триэтиламин                         & AppliChem              & 99.5\%           & ---                                 \\
            Фенилгидразин                       & Acros Organics         & 97\%             & ---                                 \\
            Хлористый бензоил                   & Реахим                 & Ч                & Предварительно перегоняли           \\
            \midrule
            Ацетон                              & ЭКОС 1                 & ЧДА              &                                     \\
            Ацетонитрил                         & Реахим                 & ЧДА              & \makecell{Перегоняли над \ce{P2O5}, \\ хранили над молекулярными\\ ситами \SI{4}{\angstrom}} \\
            Бензол                              & Реахим                 & ЧДА              & Хранили над~\ce{Na}                 \\
            Бутанол                             & Реахим                 & ЧДА              & ---                                 \\
            Гексан                              & Реахим                 & ЧДА              & ---                                 \\
            Диэтиловый эфир                     & Кузбассоргхим          & ЧДА              & ---                                 \\
            ДМФА                                & Реахим                 & ЧДА              & \makecell{Хранили над               \\молекулярными ситами \SI{4}{\angstrom} и \SI{13}{\angstrom}}\\
            ТГФ                                 & Реахим                 & ЧДА              & Хранили над~\ce{Na}                 \\
            Толуол                              & Реахим                 & ЧДА              & Хранили над~\ce{Na}                 \\
            Хлористый метилен                   & Реахим                 & ЧДА              & ---                                 \\
            Этанол                              & Реахим                 & ЧДА              & \makecell{Перегоняли над \ce{CaO},  \\ хранили над молекулярными\\ ситами \SI{4}{\angstrom}}                                 \\
            Силикагель 60-200~\si{\micro\metre} & Machery Nagel          &                  & ---                                 \\

            \bottomrule
        \end{tabular}
    \end{small}
\end{table}

Реакцию с микроволновым нагреванием проводили в микроволновом реакторе Anton~Paar Monowave~300 в режиме поддержания постоянной температуры.

Температуру плавления определяли на приборе Stuart~SMP30 или на столике Кофлера.

Спектральные данные получены в Исследовательском химическом центре коллективного пользования СО~РАН.
Спектры ЯМР регистрировали на спектрометрах Bruker~AV-300~(\ce{^1H}, \SI{300.13}{\mega\hertz}; \ce{^19F},~\SI{282.37}{\mega\hertz}) и Bruker~AV-400~(\ce{^1H},~\SI{400.13}{\mega\hertz}) в дейтерохлороформе, \mbox{ацетонитриле-\ce{d3}} и \mbox{ацетоне-\ce{d6}}.
Значения химических сдвигов протонов приведены относительно сигналов остаточных протонов растворителей~(\chemdelta{}\textsubscript{H} = 7.26, 1.94, 2.05 \si{\ppm} соответственно).
При регистрации спектров ЯМР~\ce{^19F} в качестве внутреннего стандарта использовали~\ce{C6F6}~(\chemdelta{}\textsubscript{F} = \SI{0}{\ppm}).
% Спектры ЯМР~\ce{^13C} регистрировали в режиме широкополосной развязки~(broadband decoupling, BB).

Электронные спектры поглощения регистрировали на спектрофотометре Hewlett~Packard~8453, спектры флуоресценции~--- на спектрофлуориметре Cary~Eclipse~(Varian).

Масс-спектры для соединений с молекулярными массами до \SI{800}{\dalton} получены на масс-спектрометре высокого разрешения Thermo Electron~DFS~GC-MS~(США) в условиях прямого ввода с ионизирующим напряжением \SI{70}{\electronvolt}.
Масс­-спектры высокого разрешения соединений с массой более \SI{1000}{\dalton} получены на приборе Q~Exactive HF~Thermo Fisher Scientifiс~(США) в режиме прямого анализа без хроматографического разделения\footnote{Исследование выполнено в центре масс-спектрометрического анализа ИХБФМ СО РАН}.
Анализ проведен в режиме изократического элюирования \SI{50}{\percent} метанолом, содержащим \SI{0.1}{\percent} муравьиной кислоты. Масс-спектр получен  в режиме полного сканирования с разрешением \num{120000}, диапазон сканирования варьировался в зависимости от массы соединения, AGS target-10\textsuperscript{6}.

Альдегид~\cmpd{decafluoropyrazoline_CHO} синтезировали по~\cite{2016a,2010,Filler1975}. Дицианоизофорон~\cmpd{dcif} получали по~\cite{Lemke1974}.

\textbf{\iupac{4-\{3,5-Бис[2,3,5,6-тетрафтор-4-(4-гидроксипиперидин-1-ил)фенил]-4,5-дигидро-1\H-пиразол-1-ил\}бензальдегид}~(\cmpd{decafluoropyrazoline_substituted.piperidine}).}\todo{R19}
Раствор \SI{3.00}{\gram}~(\SI{6.30}{\mmol}) альдегида~\cmpd{decafluoropyrazoline_CHO} и \SI{1.80}{\gram}~(\SI{17.80}{\mmol}) 4-гидроксипиперидина в \SI{50}{\milli\litre} сухого \ac{dmf} нагревали до \SI{100}{\celsius}, выдерживали при этой температуре 6 часов и оставляли на ночь.
Реакционную смесь выливали в \SI{400}{\milli\litre} воды со льдом, перемешивали до таяния льда и отфильтровывали осадок.
Осадок на фильтре промывали водой до нейтральной реакции, затем гексаном и сушили на воздухе.
Желто-оранжевый порошок, выход~\SI{3.70}{\gram}~--- смесь~\cmpd{decafluoropyrazoline_substituted.piperidine} и~\cmpd{decafluoropyrazoline_substituted.Me2N}.
Продукт очищали колоночной хроматографией на \ce{SiO2}, элюент~--- \ce{CH2Cl2}\,:\,ацетонитрил, градиент 5:1~-- 2:3. Собирали желтые фракции, анализировали ТСХ (\ce{CH2Cl2}\,:\,ацетонитрил, 2:1, R\textsubscript{f} $\approx$ 0.25\,--\,0.3).
Желтый порошок, выход~\cmpd{decafluoropyrazoline_substituted.piperidine} \SI{2.00}{\gram}~(\SI{52}{\percent}).
\begin{experimental}[]
    \data*{Т\textsubscript*{пл.}} 155--\SI{159}{\celsius}.
    \data{ЯМР~\ce{^1H}}[ацетон­-d\textsubscript{6}]~\chemdelta{},~\si{\ppm}: 9.77~(с, \ce{1H}, \ce{CHO}), 7.76~(д, \ce{2H_{Ar}}, \textit{J}\,=\,8.8~\si{\hertz}), 7.17~(д, \ce{2H_{Ar}}, \textit{J}\,=\,8.8~\si{\hertz}), 5.98~(дд,~\ce{1H_{pyr}},~\textit{J}\,=\,13.0, 5.3~\si{\hertz}), 4.16~(дд,~\ce{1H_{pyr}},~\textit{J}\,=\,18.2, 13.0~\si{\hertz}), 3.93~(д,~\ce{1H}, \ce{OH}, \textit{J}\,=\,4.3~\si{\hertz}), 3.83~(д,~\ce{1H}, \ce{OH}, \textit{J}\,=\,4.3~\si{\hertz}, 3.86\,--\,3.69~(м, \ce{2H}, \ce{2CH_{pip}}), 3.63\,--\,3.47~(м, \ce{3H}, \ce{CH2_{pip}}, \ce{1H_{pyr}}), 3.47\,--\,3.31~(м, \ce{2H}, \ce{CH2_{pip}}), 3.28\,--\,3.17~(м, \ce{2H}, \ce{CH2_{pip}}), 3.17\,--\,3.03~(м, \ce{2H}, \ce{CH2_{pip}}).
    \data{ЯМР~\ce{^19F}}[ацетон-\ce{d6}] \chemdelta,~\si{\ppm}: 22.27, 18.05, 12.62, 11.67~(1:1:1:1).
    \data{МС:~} Найдено \ce{[M+]}: \num{668.2023}. \ce{C32H28F8N4O3}. Рассчитано: \ce{M} \num{668.2028}.
    % \data{ЯМР~\ce{^13C}}[\ce{CDCl3}] \chemdelta, м.д.: 190.41, 147.40, 146.52, 146.02, 144.49, 144.05, 142.98, 141.03, 140.61, 131.67, 131.03, 130.93, 128.59, 112.66, 110.35, 110.22, 110.10, 104.00, 77.15, 76.90, 76.64, 67.18, 67.15, 51.57, 48.54, 48.51, 48.48, 48.44, 43.57, 34.85, 34.83.
    % \data{ЯМР~\ce{^1H}}[ацетон-d\textsubscript{6}]~\chemdelta{},~\si{\ppm}: 9.77~(с,~\ce{1H}, \ce{CHO}), 7.76~(д,~\ce{2H_{Ar}},~\textit{J}~=~8.8~\si{\hertz}), 7.17~(д,~\ce{2H_{Ar}},~\textit{J}~=~8.8~\si{\hertz}), 5.98~(дд,~\ce{1H_{pyr}},~\textit{J}~=~13.0, 5.3~\si{\hertz}), 4.16~(дд,~\ce{1H_{pyr}},~\textit{J}~=~18.2, 13.1~\si{\hertz}), 3.95~--~3.69~(м,~\ce{3H}, \ce{1H_{pyr}}, \ce{2CH_{pip}}), 3.61~--~3.46~(м,~\ce{4H}, \ce{2CH2_{pip}}), 3.30~–-~3.02~(м,~\ce{4H}, \ce{2CH2_{pip}}), 2.01~--~1.81~(м,~\ce{4H}, \ce{2CH2_{pip}}), 1.75~--~1.52~(м,~\ce{4H}, \ce{2CH2_{pip}}).
\end{experimental}

\textbf{\iupac{4-­\{3-(4-­Диметиламино-2,3,5,6­-тетрафторфенил)-5-[4-(4-гидроксипиперидин-1-ил)-2,3,5,6­-тетрафторфенил]-4,5-дигидро-1\H-пиразол­-1-­ил\}бензальдегид}~(\cmpd{decafluoropyrazoline_substituted.Me2N}).}
Желтый порошок, выход~\SI{0.41}{\gram}~(\SI{11}{\percent}).
\begin{experimental}[]
    \data*{Т\textsubscript*{пл.}} \SI{180}{\celsius}.
    \data{ЯМР~\ce{^1H}}[ацетон­-d\textsubscript{6}]~\chemdelta{},~\si{\ppm}: 9.77~(с, \ce{1H}, \ce{CHO}), 7.76~(д, \ce{2H_{Ar}}, \textit{J}\,=\,8.6~\si{\hertz}), 7.17~(д, \ce{2H_{Ar}}, \textit{J}\,=\,8.6~\si{\hertz}), 5.97(дд,~\ce{1H_{pyr}},~\textit{J}\,=\,13.1, 5.1~\si{\hertz}), 4.15(дд,~\ce{1H_{pyr}},~\textit{J}\,=\,18.2, 5.1~\si{\hertz}), 3.93~(д,~\ce{1H}, \ce{OH}, \textit{J}\,=\,4.0~\si{\hertz}), 3.88\,--\,3.77~(м, \ce{1H}, \ce{CH_{pip}}), 3.59\,--\,3.46~(м, \ce{3H}, \ce{CH2_{pip}}, \ce{1H_{pyr}}), 3.29\,--\,3.15~(м, \ce{2H}, \ce{CH2_{pip}}), 2.92~(с, \ce{6H}, \ce{NMe2}), 1.96\,--\,1.80~(м, \ce{2H}, \ce{CH2_{pip}}), 1.72\,--\,1.57~(м, \ce{2H}, \ce{CH2_{pip}}).
    \data{ЯМР~\ce{^19F}}[ацетон-\ce{d6}] \chemdelta,~\si{\ppm}: 22.24, 17.86, 11.95, 11.68~(1:1:1:1).
    \data{МС:~} Найдено \ce{[M+]}: \num{612.1758}. \ce{C29H24F8N4O2}. Рассчитано: \ce{M} \num{612.1766}.
\end{experimental}

\textbf{Взаимодействие соединения~\cmpd{decafluoropyrazoline_substituted.piperidine} с недостатком 4-гидроксипиперидина.}
Раствор~\SI{0.10}{\gram}~(\SI{0.20}{\mmol}) альдегида~\cmpd{decafluoropyrazoline_substituted.piperidine} и \SI{0.03}{\gram}~(\SI{0.30}{\mmol}) 4-гидроксипиперидина в \SI{2}{\ml} сухого ДМФА нагревали с перемешиванием при \SI{60}{\celsius} 3 часа, оставляли на ночь. Наблюдается ярко-оранжевое окрашивание. Выливали на лед, выпавший осадок отфильтровывали и анализировали с помощью спектров ЯМР~\ce{^1H} и \ce{^19F}. Основным продуктом реакции является альдегид, замещенный гидроксипиперидиногруппой в одном из фторированных колец, наряду с дизамещенным альдегидом; соотношение 4:1. Данные спектра ЯМР~\ce{^19F} указывают на то, что незамещенным остается пентафторфенильное кольцо в положении 3 пиразолинового цикла, в котором отсутствует характерное уширение сигнала \emph{орто}-атомов фтора.

% \textbf{Взаимодействие соединения~\cmpd{decafluoropyrazoline_substituted.piperidine} с пиперазином.}
% Раствор~\SI{1.00}{\gram}~(\SI{1.98}{\mmol}) альдегида~\cmpd{decafluoropyrazoline_substituted.piperidine} и \SI{1.70}{\gram}~(\SI{19.8}{\mmol}) пиперазина в \SI{15}{\ml} сухого ДМФА нагревали с перемешиванием при \SI{85}{\celsius} 8 часов. После охлаждения до комнатной температуры смесь  выливали в воду со льдом, выпавший ярко-желтый осадок отфильтровывали, промывали водой от избытка пиперазина и сушили на воздухе. Вес \SI{0.98}{\gram}. По данным спектров ЯМР~\ce{^1H} и \ce{^19F} продукт предположительно представляет собой смесь двух соединений ˗целевого бис(4-пиперазинил)замещенного альдегида и соединения, содержащего в одном из тетрафторфенильных колец пиперазиногруппу, а в другом~--- диметиламиногруппу, в соотношении 84:16. Разделить соединения не удалось.

\textbf{\iupac{\E-2-[3-(4-\{3,5-Бис[2,3,5,6-тетрафтор-4-(4-гидроксипиперидин-1-ил)фенил]-4,5-дигидро-1\H-пиразол-1-ил\}стирил)-5,5-диметилциклогекс-2-ен-1-илиден]малононитрил}~(\cmpd{decafluoropyrazoline_DCIF.piperidine}).}\todo{R23} К раствору \SI{0.25}{\gram}~(\SI{0.37}{\mmol}) альдегида~\textbf{\cmpd{decafluoropyrazoline_substituted.piperidine}} и \SI{0.07}{\gram}~(\SI{0.37}{\mmol}) дицианоизофорона в \SI{5}{\milli\litre} бутанола прибавляли 5 капель морфолина, кипятили в атмосфере аргона 7 часов и оставляли на ночь. Выпавший осадок отфильтровывали, промывали этанолом и диэтиловым эфиром. Темно-красный порошок, выход~\SI{0.13}{\gram}~(\SI{42}{\percent}).
\begin{experimental}
    \data*{Т\textsubscript*{пл.}} 239--\SI{241}{\celsius}.
    \data{ЭСП}[ацетон] \chemlambda\textsubscript{max}~($\lg \varepsilon$): \SI{501}{\nano\metre}~(4.47).
    \data{ЯМР~\ce{^1H}}[\ce{CDCl3}]~\chemdelta,~\si{\ppm}: 7.37~(д,~\ce{2H_{Ar}},~\textit{J}\,=\,8.7~\si{\hertz}), 7.05~(д,~\ce{2H_{Ar}},~\textit{J}\,=\,8.7~\si{\hertz}), 6.96~(д,~\ce{1H}, \ce{CH=},~\textit{J}\,=\,15.9~\si{\hertz}), 6.80~(д,~\ce{1H}, \ce{=CH},~\textit{J}\,=\,15.9~\si{\hertz}), 6.73~(с,~\ce{1H}, \ce{=CH_{isoph}}), 5.71~(дд,~\ce{1H_{pyr}},~\textit{J}\,=\,13.0, 5.6~\si{\hertz}), 3.99\,--\,3.76~(м,~\ce{3H}, \ce{2CH_{pip}}, \ce{1H_{pyr}}), 3.75\,--\,3.60~(м,~\ce{2H}, \ce{2OH_{pip}}), 3.55\,--\,3.44~(м,~\ce{2H}, \ce{CH2_{pip}}), 3.44\,--\,3.31~(м,~\ce{3H}, \ce{CH2_{pip}}, \ce{1H_{pyr}}), 3.25\,--\,3.02~(м,~\ce{4H}, \ce{2CH2_{pip}}), 2.55~(с,~\ce{2H}, \ce{CH2_{isoph}}), 2.41~(с,~\ce{2H}, \ce{CH2_{isoph}}), 1.97~(м,~\ce{4H}, \ce{2CH2_{pip}}), 1.78\,--\,1.63~(м,~\ce{4H}, \ce{2CH2_{pip}}), 1.04~(с,~\ce{6H}, \ce{2CH3_{isoph}}).
    \data{ЯМР~\ce{^19F}}[\ce{CDCl3}]~\chemdelta,~\si{\ppm}: 20.65, 16.69, 11.44, 10.11~(1:1:1:1).
    \data{МС:~} Найдено \ce{[M+]}: \num{836.3075}. \ce{C44H40F8N6O2}. Рассчитано: \ce{M} \num{836.3080}.
\end{experimental}

\textbf{\iupac{\{[1-(4-Формилфенил)-4,5-дигидро-1\H-пиразол-3,5-диил]бис(2,3,5,6-тетрафтор-4,1-фенилен)\}бис(пиперидин-1,4-диил)дибензоат}~(\cmpd{decafluoropyrazoline_piperidine_benzoyl}).}\todo{R20}
\textbf{Способ 1.} Суспензию \SI{0.50}{\gram}~(\SI{0.75}{\mmol}) альдегида~\textbf{\cmpd{decafluoropyrazoline_substituted.piperidine}} в \SI{10}{\milli\litre} сухого бензола доводили до кипения и прибавляли к ней \SI{0.62}{\milli\litre}~(\SI{4.50}{\mmol}) триэтиламина и \SI{0.35}{\milli\litre}~(\SI{3.00}{\mmol}) хлористого бензоила. После двух часов кипячения прибавляли еще столько же триэтиламина и хлористого бензоила и кипятили еще сутки. Реакционную смесь выливали в \SI{100}{\milli\litre} воды и добавляли бензол до разделения фаз. Органическую фазу отделяли, сушили над \ce{Na2SO4} и удаляли растворитель в вакууме. Твердый остаток очищали колоночной хроматографией на \ce{SiO2}, элюент~--- бензол\,:\,\ce{CHCl3}, градиент 1:0\,--\,0:1. Собирали желтые фракции, элюент удаляли в вакууме и повторно очищали колоночной хроматографией на \ce{SiO2}, элюент~--- смесь бензол\,:\,\ce{CH2Cl2}~1:1. Собирали желтые фракции, растворитель удаляли в вакууме. Желтое масло, выход~\SI{0.49}{\gram}~(\SI{74}{\percent}).

\textbf{Способ 2.} К суспензии \SI{0.20}{\gram}~(\SI{0.30}{\mmol}) альдегида~\textbf{\cmpd{decafluoropyrazoline_substituted.piperidine}} в \SI{5}{\milli\litre} сухого бензола, прибавляли~\SI{0.11}{\milli\litre}~(\SI{0.90}{\mmol}) хлористого бензоила, \SI{0.13}{\milli\litre}~(\SI{0.90}{\mmol}) триэтиламина и \SI{2}{\milli\gram}~\ac{dmap}.
Реакционную смесь кипятили 6 часов, оставляли на ночь и удаляли растворитель в вакууме.
Полученное масло очищали колоночной хроматографией на \ce{SiO2}, элюент~--- смесь ацетонитрил\,:\,\ce{CH2Cl2}, градиент 1:1 -- 8:1, собирали желтую фракцию, элюент удаляли в вакууме, полученное масло промывали смесью гексана с диэтиловым эфиром 1:1. Светло-желтый порошок, выход~\SI{0.19}{\gram}~(\SI{74}{\percent}).
\begin{experimental}
    \data*{Т\textsubscript*{пл.}} 180--\SI{183}{\celsius}.
    \data{ЯМР~\ce{^1H}}[\ce{CDCl3}]~\chemdelta,~м.д.: 9.77~(с,~\ce{1H},~\ce{CHO}), 8.00 -- 8.14 ~(м,~\ce{4H_{Ar}}), 7.73~(д,~\ce{2H_{Ar}},~\textit{J}\,=\,8.4~\si{\hertz}), 7.61\,--\,7.52~(м,~\ce{2H_{Ar}}), 7.50\,--\,7.39~(м,~\ce{4H_{Ar}}), 7.13~(д,~\ce{2H_{Ar}},~\textit{J}\,=\,8.4~\si{\hertz}), 5.75(дд,~\ce{1H_{pyr}},~\textit{J}\,=\,13.0, 5.9~\si{\hertz}), 3.95~(дд,~\ce{1H_{pyr}},~\textit{J}\,=\,17.8, 5.9~\si{\hertz}), 5.35 -- 5.11~(м,~\ce{3H}, \ce{2CH_{pip}}, \ce{1H_{pyr}}), 3.65\,--\,3.41~(м,~\ce{4H}, \ce{2CH2_{pip}}), 3.41\,--\,3.13~(м,~\ce{4H}, \ce{2CH2_{pip}}), 2.26\,--\,2.03~(м,~\ce{4H}, \ce{2CH2_{pip}}), 2.03\,--\,1.87~(м,~\ce{4H}, \ce{2CH2_{pip}}).
    \data{ЯМР~\ce{^19F}}[\ce{CDCl3}]~\chemdelta,~\si{\ppm}: 21.14, 16.72, 11.74, 10.43~(1:1:1:1).
    \data{МС:~} Найдено \ce{[M+]}: \num{876.2548}. \ce{C46H36F8N4O5}. Рассчитано: \ce{M} \num{876.2553}.
\end{experimental}

\textbf{\iupac{\E-\{[1-(4-\{2-[3-(Дицианометилен)-5,5-диметилциклогекс-1-ен-1-ил]винил\}фенил)-4,5-дигидро-1\H-пиразол-3,5-диил]бис(2,3,5,6-тетрафтор-4,1-фенилен)\}бис(пиперидин-1,4-диил)дибензоат}~(\cmpd{decafluoropyrazoline_piperidine_DCIF.benzoyl}).}\todo{R22 с двумя бензоилами}
\textbf{Способ 1.} К суспензии \SI{0.48}{\gram}~(\SI{0.55}{\mmol}) альдегида~\cmpd{decafluoropyrazoline_piperidine_benzoyl} в \SI{15}{\milli\litre} бутанола прибавляли \SI{0.10}{\gram}~(\SI{0.55}{\mmol}) дицианоизофорона и 5 капель морфолина. Смесь кипятили в атмосфере аргона 7 часов, растворитель удаляли в вакууме. Твердый остаток очищали колоночной хроматографией на \ce{SiO2}, элюент~--- \ce{CH2Cl2}\,:\,гексан, градиент 1:1\,--\,0:1, затем ацетонитрил. Собирали красные фракции, растворитель удаляли в вакууме. Темно-красный порошок, выход~\SI{0.14}{\gram}~(\SI{24}{\percent}).

\textbf{Способ 2.} К суспензии \SI{0.15}{\gram}~(\SI{0.18}{\mmol}) соединения~\textbf{\cmpd{decafluoropyrazoline_DCIF.piperidine}} в \SI{5}{\milli\litre} сухого бензола прибавляли \SI{0.07}{\milli\litre}~(\SI{0.56}{\mmol}) хлористого бензоила, \SI{0.08}{\milli\litre}~(\SI{0.56}{\mmol}) триэтиламина и \SI{1.5}{\milli\gram} \ac{dmap}.
Реакционную смесь кипятили 1 час, растворитель удаляли в вакууме.
Очищали колоночной хроматографией на \ce{SiO2}, элюент~--- смесь ацетонитрил\,:\,\ce{CH2Cl2}, градиент 0:1 -- 1:10.
Собирали красные фракции, растворитель удаляли в вакууме. Темно-красный порошок, выход~\SI{0.05}{\gram}~(\SI{25}{\percent}).
\begin{experimental}
    \data*{Т\textsubscript*{пл.}} 116--\SI{118}{\celsius}.
    \data{ЭСП}[ацетон] \chemlambda\textsubscript{max}~($\lg \varepsilon$): \SI{490}{\nano\metre}~(4.73).
    \data{ЯМР~\ce{^1H}}[\ce{CDCl3}]~\chemdelta,~\si{\ppm}: 8.12\,--\,7.99~(м,~\ce{4H_{Ar}}), 7.61\,--\,7.51~(м,~\ce{2H_{Ar}}), 7.49\,--\,7.30~(м,~\ce{5H_{Ar}}), 7.21\,--\,7.03~(м,~\ce{3H_{Ph}}), 6.97~(д,~\ce{1H}, \ce{CH=},~\textit{J}\,=\,16.0~\si{\hertz}), 6.81~(д,~\ce{1H}, \ce{=CH},~\textit{J}\,=\,16.0~\si{\hertz}), 6.74~(с,~\ce{1H}, \ce{=CH_{isoph}}), 5.72~(дд,~\ce{1H_{pyr}},~\textit{J}\,=\,12.4, 5.8~\si{\hertz}), 5.30\,--\,5.10~(м,~\ce{2H}, \ce{2CH_{pip}}), 4.04\,--\,3.80~(м,~\ce{1H_{pyr}}), 3.61\,--\,3.16~(м,~\ce{9H}, \ce{4CH2_{pip}}, \ce{1H_{pyr}}), 2.56~(с,~\ce{2H}, \ce{CH2_{isoph}}), 2.42~(с,~\ce{2H}, \ce{CH2_{isoph}}), 2.20\,--\,1.86~(м,~\ce{8H}, \ce{4CH2_{pip}}), 1.24~(с,~\ce{6H}, \ce{2CH3_{isoph}}).
    \data{ЯМР~\ce{^19F}}[\ce{CDCl3}]~\chemdelta,~\si{\ppm}: 20.84, 16.76, 11.60, 10.19~(1:1:1:1).
    \data{МС:~} Найдено \ce{[M + H]+}: \num{1045.3609}. \ce{C58H48F8N6O4}. Рассчитано: \ce{[M + H]} \num{1045.3682}.
\end{experimental}

\textbf{\iupac{\E-\{[1-(4-\{2-[3-(дицианометилен)-5,5-диметилциклогекс-1-ен-1-ил]винил\}фенил)-1\H-пиразол-3,5-диил]бис(2,3,5,6-тетрафтор-4,1-фенилен)\}бис(пиперидин-1,4-диил)дибензоат}~(\cmpd{pyrazole}).}
\begin{experimental}
    \data{ЯМР~\ce{^1H}}[\ce{CDCl3}]~\chemdelta,~\si{\ppm}:~8.12\,--\,8.01~(м, \ce{4H_{Ar}}), 7.62\,--\,7.33 (м, \ce{10H_{Ar}}), 7.00~(дд, \ce{2H}, \ce{CH=CH}, \textit{J}\,=\,16.3~\si{\hertz}), 6.89~(с, \ce{1H}, \ce{=CH_{isoph})}, 6.84~(c, \ce{1H_{pyrazole}}), 5.31\,--\,5.18~(м, \ce{2H}, \ce{2CH_{pip}}), 3.67\,--\,3.46 (м, \ce{4H}, \ce{2CH2_{pip}}), 3.42\,--\,3.22 (м, \ce{4H}, \ce{2CH2_{pip}}), 2.59~(с, \ce{2H}, \ce{CH2_{isoph}}), 2.45~(с, \ce{2H}, \ce{CH2_{isoph}}), 2.21\,--\,2.07~(м, \ce{4H}, \ce{2CH2_{pip}}), 2.05\,--\,1.88~(м, \ce{4H}, \ce{2CH2_{pip}}), 1.07~(с, \ce{6H}, \ce{2CH3_{isoph}}).
    \data{ЯМР~\ce{^19F}}[\ce{CDCl3}]~\chemdelta,~\si{\ppm}:~21.27, 18.98, 11.47, 10.33 (1:1:1:1).
    % \data{МС:~} Найдено \ce{[M + H]+}: \num{}. \ce{C58H46F8N6O4}. Рассчитано: \ce{[M + H]} \num{}.
\end{experimental}

\textbf{Диэфиры~\cmpd{decafluoropyrazoline_piperidine_DCIF.{TAFS, TATBS}}~(общая методика).}
К раствору соединения \cmpd{decafluoropyrazoline_DCIF.piperidine} в \SI{6}{\milli\litre} сухого бензола добавляли соответствующий хлорангидрид, триэтиламин и \SI{1}{\milli\gram} \ac{dmap}.
Полученную смесь кипятили до окончания реакции.
Растворитель удаляли в вакууме, твердый остаток очищали колоночной хроматографией на \ce{SiO2}.
Элюент удаляли в вакууме, твердый продукт промывали гексаном или смесью гексан-эфир.

\textbf{\iupac{\E-\{[1-(4-\{2-[3-(Дицианометилен)-5,5-диметилциклогекс-1-ен-1-ил]винил\}фенил)-4,5-дигидро-1\H-пиразол-3,5-диил]бис(2,3,5,6-тетрафтор-4,1-фенилен)\}бис(пиперидин-1,4-диил)бис[4-метил-3,5-бис(\{[2,3,5,6-тетрафтор-4-(трифторметил)фенил]тио\}метил)бензоат]}~(\cmpd{decafluoropyrazoline_piperidine_DCIF.TAFS}).} \todo{пиперидин с двумя TAFS R30}
Получен по общей методике из \SI{0.10}{\gram}~(\SI{0.12}{\milli\mole}) соединения~\cmpd{decafluoropyrazoline_DCIF.piperidine}, \SI{0.24}{\gram}~(\SI{0.36}{\milli\mole}) \ce{TAFS-Cl}, \SI{0.10}{\milli\litre}~(\SI{0.72}{\milli\mole}) триэтиламина и \SI{1}{\milli\gram}~\ac{dmap}.
Время реакции 2 часа. Твердый остаток после удаления растворителя очищали колоночной хроматографией на \ce{SiO2}, элюент~--- \ce{CH2Cl2}.
Темно-красный порошок, выход~\SI{0.08}{\gram}~(\SI{30}{\percent}).
\begin{experimental}
    \data*{Т\textsubscript*{пл.}} 93--\SI{95}{\celsius}.
    \data{ЭСП}[ацетон] \chemlambda\textsubscript{max}~($\lg \varepsilon$): \SI{491}{\nano\metre}~(4.73).
    \data{ЯМР~\ce{^1H}}[\ce{CD3CN}]~\chemdelta,~\si{\ppm}: 7.68\,--\,7.61~(м,~\ce{4H_{TAFS}}), 7.57~(д,~\ce{2H_{Ar}},~\textit{J}\,=\,8.6~\si{\hertz}), 7.22~(д,~\ce{1H}, \ce{CH=},~\textit{J}\,=\,16.2~\si{\hertz}), 7.15~(д,~\ce{2H_{Ar}},~\textit{J}\,=\,8.6~\si{\hertz}), 7.09~(д,~\ce{1H}, \ce{=CH},~\textit{J}\,=\,16.2~\si{\hertz}), 6.87~(с,~\ce{1H}, \ce{=CH_{isoph}}), 5.95~(дд,~\ce{1H_{pyr}},~\textit{J}\,=\,11.3, 3.4~\si{\hertz}), 5.28\,--\,5.09~(м,~\ce{2H}, \ce{2CH_{pip}}), 4.49\,--\,4.37~(м,~\ce{8H}, \ce{4CH2_{TAFS}}), 4.24\,--\,3.99~(м,~\ce{1H_{pyr}}), 3.73\,--\,3.23~(м,~\ce{9H}, \ce{1H_{pyr}}, \ce{4CH2_{pip}}), 2.60~(с,~\ce{2H}, \ce{CH2_{isoph}}), 2.21~(с,~\ce{6H}, \ce{2CH3_{TAFS}}), 2.18\,--\,2.11~(м,~\ce{4H}, \ce{2CH2_{pip}}), 2.00\,--\,1.80~(м,~\ce{4H}, \ce{2CH2_{pip}}), 1.45\,--\,1.34~(м,~\ce{2H}, \ce{CH2_{isoph}}), 1.14~(с,~\ce{6H}, \ce{2CH3_{isoph}}).
    \data{ЯМР~\ce{^19F}}[\ce{CD3CN}]~\chemdelta,~\si{\ppm}: 107.27, 32.00, 21.82, 17.84, 12.52, 11.34~(6:4:5:1:1).
    \data{МС:~} Найдено \ce{[M - H]-}: \num{2119.2502}. \ce{C92H56F36N6O4S4}. Рассчитано: \ce{[M - H]} \num{2119.2598}.
\end{experimental}

\textbf{\iupac{\E-\{[1-(4-\{2-[3-(Дицианометилен)-5,5-диметилциклогекс-1-ен-1-ил]винил\}фенил)-4,5-дигидро-1\H-пиразол-3,5-диил]бис(2,3,5,6-тетрафтор-4,1-фенилен)\}бис(пиперидин-1,4-диил)бис[3,5-бис(\{[4-(\textit{трет}-бутил)фенил]тио\}метил)-4-метилбензоат]}~(\cmpd{decafluoropyrazoline_piperidine_DCIF.TATBS}).}\todo{пиперидин с двумя TATBS R33}
Получен по общей методике из \SI{0.06}{\gram}~(\SI{0.07}{\milli\mole}) соединения~\cmpd{decafluoropyrazoline_DCIF.piperidine}, \SI{0.12}{\gram}~(\SI{0.22}{\milli\mole}) \ce{TATBS-Cl}, \SI{0.14}{\milli\litre}~(\SI{1.10}{\milli\mole}) триэтиламина и \SI{1}{\milli\gram}~\ac{dmap}.
Время реакции 6 часов. Твердый остаток после удаления растворителя очищали колоночной хроматографией на \ce{SiO2}, элюент~--- бензол.
Темно-красный порошок, выход~\SI{0.07}{\gram}~(\SI{55}{\percent}). \todo{ЭСП}
\begin{experimental}
    \data{ЯМР~\ce{^1H}}[ацетон-d\textsubscript{6}]~\chemdelta,~\si{\ppm}: 7.78~(с,~\ce{2H_{TATBS}}), 7.75~(с,~\ce{2H_{TATBS}}), 7.60~(д,~\ce{2H_{Ar}},~\textit{J}\,=\,8.7~\si{\hertz}), 7.37\,--\,7.25~(м,~\ce{16H_{TATBS}}), 7.20~(д,~\ce{2H_{Ar}},~\textit{J}\,=\,7.3~\si{\hertz}), 7.13\,--\,7.05~(м,~\ce{2H}, \ce{CH=CH}), 6.78~(с,~\ce{1H}, \ce{=CH_{isoph}}), 5.98~(дд,~\ce{1H_{pyr}},~\textit{J}\,=\,13.0, 5.6~\si{\hertz}), 5.21\,--\,5.04~(м,~\ce{2H}, \ce{2CH_{pip}}), 4.27~(с,~\ce{4H}, \ce{2CH2_{TATBS}}), 4.25~(с,~\ce{4H}, \ce{2CH2_{TATBS}}), 4.14~(дд,~\ce{1H_{pyr}},~\textit{J}\,=\,18.0, 13.0~\si{\hertz}), 3.65\,--\,3.21~(м,~\ce{9H}, \ce{4CH2_{pip}}, \ce{1H_{pyr}}), 2.62~(с,~\ce{2H}, \ce{CH2_{isoph}}), 2.59\,--\,2.56~(м,~\ce{2H}, \ce{CH2_{isoph}}), 2.50~(с,~\ce{3H}, \ce{CH3_{TATBS}}), 2.48~(с,~\ce{3H}, \ce{CH3_{TATBS}}), 2.14\,--\,2.07~(м,~\ce{4H}, \ce{2CH2_{pip}}), 1.99\,--\,1.75~(м,~\ce{4H}, \ce{2CH2_{pip}}), 1.27~(с,~\ce{18H}, \ce{tBu_{TATBS}}), 1.25~(с,~\ce{18H}, \ce{tBu_{TATBS}}), 1.07~(с,~\ce{6H}, \ce{2CH3_{isoph}}).
    \data{ЯМР~\ce{^19F}}[ацетон-d\textsubscript{6}]~\chemdelta,~\si{\ppm}: 22.23, 18.47, 12.75, 11.83~(1:1:1:1).
    \data{МС:~} Найдено \ce{[M + H]+}: \num{1785.7141}. \ce{C104H108F8N6O4S4}. Рассчитано: \ce{[M + H]} \num{1785.7260}.
\end{experimental}

\textbf{Моноэфиры~\cmpd{pentafluoropyrazoline_piperidine_DCIF.{benzoyl, TAFS, TATBS, IDATBS}}~(общая методика).}
К раствору \SI{0.10}{\gram}~(\SI{0.15}{\milli\mole}) соединения \cmpd{pentafluoropyrazoline_DCIF.piperidine} в \SI{6}{\milli\litre} сухого бензола добавляли 0.2-0.3~\si{\milli\mole} соответствующего хлорангидрида , 0.2-0.3~\si{\milli\mole} триэтиламина и \SI{1}{\milli\gram} \ac{dmap}.
Полученную смесь кипятили до окончания реакции.
Растворитель удаляли в вакууме, твердый остаток очищали колоночной хроматографией на \ce{SiO2}, элюент~--- бензол.
Элюент удаляли в вакууме, твердый продукт промывали гексаном или смесью гексан-эфир.

\textbf{\iupac{\E-1-\{4-[1-(4-\{2-[3-(Дицианометилен)-5,5-диметилциклогекс-1-ен-1-ил]винил\}фенил)-3-фенил-4,5-дигидро-1\H-пиразол-5-ил]-2,3,5,6-тетрафторфенил\}пиперидин-4-илбензоат}~(\cmpd{pentafluoropyrazoline_piperidine_DCIF.benzoyl}).} \todo{пиперидин с одним бензоилом}
Получен по общей методике из \SI{0.10}{\gram}~(\SI{0.15}{\milli\mole}) соединения~\cmpd{pentafluoropyrazoline_DCIF.piperidine}, \SI{0.03}{\milli\litre}~(\SI{0.23}{\milli\mole}) хлористого бензоила, \SI{0.04}{\milli\litre}~(\SI{0.23}{\milli\mole}) триэтиламина и \SI{1}{\milli\gram}~\ac{dmap}.
Время реакции 4 часа.
Темно-красный порошок, выход~\SI{0.11}{\gram}~(\SI{92}{\percent}).
\begin{experimental}
    \data*{Т\textsubscript*{пл.}} 145--\SI{147}{\celsius}.
    \data{ЭСП}[ацетон] \chemlambda\textsubscript{max}~($\lg \varepsilon$): \SI{499}{\nano\metre}~(4.68).
    \data{ЯМР~\ce{^1H}}[ацетон-d\textsubscript{6}]~\chemdelta,~\si{\ppm}: 8.03~(д,~\ce{2H_{Ar}},~\textit{J}\,=\,7.6~\si{\hertz}), 7.82~(д,~\ce{2H_{Ar}},~\textit{J}\,=\,7.6~\si{\hertz}), 7.67\,--\,7.54~(м,~\ce{3H_{Ph}}), 7.54\,--\,7.36~(м,~\ce{5H_{Ph}}), 7.22~(д,~\ce{1H}, \ce{CH=},~\textit{J}\,=\,15.3~\si{\hertz}), 7.18\,--\,7.04~(м,~\ce{3H}, \ce{=CH}, \ce{2H_{Ph}}), 6.76~(с,~\ce{1H}, \ce{=CH_{isoph}}), 5.94~(дд,~\ce{1H_{pyr}},~\textit{J}\,=\,13.0, 5.6~\si{\hertz}), 5.28\,--\,5.06~(м,~\ce{1H}, \ce{CH_{pip}}), 4.08~(дд,~\ce{1H_{pyr}},~\textit{J}\,=\,18.0, 13.0~\si{\hertz}), 3.65\,--\,3.40~(м,~\ce{3H}, \ce{1H_{pyr}}, \ce{CH2_{pip}}), 3.28~(с,~\ce{2H}, \ce{CH2_{pip}}), 2.61~(с,~\ce{2H}, \ce{CH2_{isoph}}), 2.56~(с,~\ce{2H}, \ce{CH2_{isoph}}), 1.94\,--\,1.75~(м,~\ce{4H}, \ce{2CH2_{pip}}), 1.06~(с,~\ce{6H}, \ce{2CH3_{isoph}}).
    \data{ЯМР~\ce{^19F}}[ацетон-d\textsubscript{6}]~\chemdelta,~\si{\ppm}: 18.23, 12.70~(1:1).
    \data{МС:~} Найдено \ce{[M+]}: \num{769.3036}. \ce{C46H39F4N5O2}. Рассчитано: \ce{M} \num{769.3034}.
\end{experimental}

\textbf{\iupac{\E-1-\{4-[1-(4-\{2-[3-(Дицианометилен)-5,5-диметилциклогекс-1-ен-1-ил]винил\}фенил)-3-фенил-4,5-дигидро-1\H-пиразол-5-ил]-2,3,5,6-тетрафторфенил\}пиперидин-4-ил-4-метил-3,5-бис(\{[2,3,5,6-тетрафтор-4-(трифторметил)фенил]тио\}метил)бензоат}~(\cmpd{pentafluoropyrazoline_piperidine_DCIF.TAFS}).} \todo{пиперидин с одним TAFS R36}
Получен по общей методике из \SI{0.10}{\gram}~(\SI{0.15}{\milli\mole}) соединения~\cmpd{pentafluoropyrazoline_DCIF.piperidine}, \SI{0.15}{\gram}~(\SI{0.23}{\milli\mole}) \ce{TAFS-Cl}, \SI{0.04}{\milli\litre}~(\SI{0.23}{\milli\mole}) триэтиламина и \SI{1}{\milli\gram}~\ac{dmap}.
Время реакции 2.5 часа.
Темно-красный порошок, выход~\SI{0.19}{\gram}~(\SI{97}{\percent}).
\begin{experimental}
    \data*{Т\textsubscript*{пл.}} 108--\SI{111}{\celsius}.
    \data{ЭСП}[ацетон] \chemlambda\textsubscript{max}~($\lg \varepsilon$): \SI{499}{\nano\metre}~(4.61).
    \data{ЯМР~\ce{^1H}}[ацетон-d\textsubscript{6}]~\chemdelta,~\si{\ppm}: 7.84~(д,~\ce{2H_{Ar}},~\textit{J}\,=\,8.8~\si{\hertz}), 7.65~(с,~\ce{2H_{TAFS}}), 7.58~(д,~\ce{2H_{Ar}},~\textit{J}\,=\,8.8~\si{\hertz}), 7.50\,--\,7.38~(м,~\ce{3H_{Ph}}), 7.34~(с,~\ce{1H_{Ph}}), 7.28\,--\,7.04~(м,~\ce{3H}, \ce{1H_{Ph}}, \ce{CH=CH}), 6.77~(с,~\ce{1H}, \ce{=CH_{isoph}}), 5.95~(дд,~\ce{1H_{pyr}},~\textit{J}\,=\,12.9, 5.7~\si{\hertz}), 5.17\,--\,5.00~(м,~\ce{1H}, \ce{CH_{pip}}), 4.47~(с,~\ce{4H}, \ce{2CH2_{TAFS}}), 4.09~(дд,~\ce{1H_{pyr}},~\textit{J}\,=\,17.8, 12.9~\si{\hertz}), 3.64\,--\,3.18~(м,~\ce{5H}, \ce{1H_{pyr}}, \ce{2CH2_{pip}}), 2.66~(с,~\ce{3H}, \ce{CH3_{TAFS}}), 2.61~(с,~\ce{2H}, \ce{CH2_{isoph}}), 2.56~(с,~\ce{2H}, \ce{CH2_{isoph}}), 1.83~(с,~\ce{2H}, \ce{CH2_{pip}}), 1.30\,--\,1.22~(м,~\ce{2H}, \ce{CH2_{pip}}), 1.07~(с,~\ce{6H}, \ce{2CH3_{isoph}}).
    \data{ЯМР~\ce{^19F}}[ацетон-d\textsubscript{6}]~\chemdelta,~\si{\ppm}: 107.51, 32.06, 21.84, 18.23, 12.65~(3:2:2:1:1).
    \data{МС:~} Найдено \ce{[M + H]+}: \num{1308.2568}. \ce{C63H43F18N5O2S2}. Рассчитано: \ce{[M + H]} \num{1308.2644}.
\end{experimental}

\textbf{\iupac{\E-1-\{4-[1-(4-\{2-[3-(Дицианометилен)-5,5-диметилциклогекс-1-ен-1-ил]винил\}фенил)-3-фенил-4,5-дигидро-1\H-пиразол-5-ил]-2,3,5,6-тетрафторфенил\}пиперидин-4-ил-3,5-бис(\{[4-(\textit{трет}-бутил)фенил]тио\}метил)4-метилбензоат}~(\cmpd{pentafluoropyrazoline_piperidine_DCIF.TATBS}).} \todo{пиперидин с одним TATBS R31}
\textbf{Способ 1.}~Получен по общей методике из \SI{0.10}{\gram}~(\SI{0.15}{\milli\mole}) соединения~\cmpd{pentafluoropyrazoline_DCIF.piperidine}, \SI{0.12}{\gram}~(\SI{0.23}{\milli\mole}) \ce{TATBS-Cl}, \SI{0.06}{\milli\litre}~(\SI{0.4}{\milli\mole}) триэтиламина и \SI{1}{\milli\gram}~\ac{dmap}.
Время реакции 3 часа.
Темно-красный порошок, выход~\SI{0.10}{\gram}~(\SI{59}{\percent}).

\textbf{Способ 2.}
К раствору \SI{0.05}{\gram}~(\SI{0.75}{\milli\mole}) соединения~\cmpd{pentafluoropyrazoline_DCIF.piperidine}, \SI{0.04}{\gram}~(\SI{0.75}{\milli\mole}) \ce{TATBS-OH} и \SI{0.03}{\gram}~(\SI{0.11}{\milli\mole}) трифенилфосфина в \SI{6}{\milli\litre} сухого ТГФ при перемешивании в атмосфере аргона по каплям прибавляли раствор \SI{0.02}{\milli\litre}~\ac{diad} в \SI{4}{\milli\litre} сухого ТГФ. Реакционную смесь выдерживали при комнатной температуре 2.5 часа, растворитель удаляли в вакууме. Твердый остаток очищали колоночной хроматографией на \ce{SiO2}, элюент~--- бензол. Собирали красные фракции, элюент удаляли в вакууме. Темно-красный порошок, выход~\SI{0.06}{\gram}~(\SI{70}{\percent}).

\textbf{Способ 3.}
Раствор \SI{0.05}{\gram}~(\SI{0.75}{\milli\mole}) соединения~\cmpd{pentafluoropyrazoline_DCIF.piperidine}, \SI{0.04}{\gram}~(\SI{0.75}{\milli\mole}) \ce{TATBS-OH}, \SI{0.02}{\gram}~(\SI{0.83}{\milli\mole})~\ac{dcc} и \SI{1}{\milli\gram}~\ac{dmap} в \SI{6}{\milli\litre} сухого бензола выдерживали 5 часов, после кипятили 7 часов. Растворитель удаляли в вакууме, Твердый остаток очищали колоночной хроматографией на \ce{SiO2}, элюент~--- бензол. Собирали красные фракции, элюент удаляли в вакууме. Темно-красный порошок, выход~\SI{0.02}{\gram}~(\SI{22}{\percent}).
\begin{experimental}
    \data*{Т\textsubscript*{пл.}} 108--\SI{110}{\celsius}.
    \data{ЭСП}[ацетон] \chemlambda\textsubscript{max}~($\lg \varepsilon$): \SI{499}{\nano\metre}~(4.67).
    \data{ЯМР~\ce{^1H}}[ацетон-d\textsubscript{6}]~\chemdelta,~\si{\ppm}: 7.82~(д,~\ce{2H_{Ar}},~\textit{J}\,=\,8.5~\si{\hertz}), 7.72~(с,~\ce{2H_{TATBS}}), 7.57~(д,~\ce{2H_{Ar}},~\textit{J}\,=\,8.5~\si{\hertz}), 7.48\,--\,7.37~(м,~\ce{3H_{Ph}}), 7.32~(с,~\ce{2H_{Ph}}), 7.30\,--\,7.21~(м,~\ce{8H_{TATBS}}), 7.20\,--\,7.00~(м,~\ce{2H}, \ce{CH=CH}), 6.75~(с,~\ce{1H}, \ce{=CH_{isoph}}), 5.94~(дд,~\ce{1H_{pyr}},~\textit{J}\,=\,12.9, 5.6~\si{\hertz}), 5.06~(м,~\ce{1H}, \ce{CH_{pip}}), 4.23~(с,~\ce{4H}, \ce{2CH2_{TATBS}}), 4.07~(дд,~\ce{1H_{pyr}},~\textit{J}\,=\,17.8, 12.9~\si{\hertz}), 3.62\,--\,3.38~(м,~\ce{3H}, \ce{1H_{pyr}}, \ce{CH2_{pip}}), 3.34\,--\,3.16~(м,~\ce{2H}, \ce{CH2_{pip}}), 2.60~(с,~\ce{2H}, \ce{CH2_{isoph}}), 2.55~(с,~\ce{2H}, \ce{CH2_{isoph}}), 2.47~(с,~\ce{3H}, \ce{CH3_{TATBS}}), 1.90\,--\,1.73~(м,~\ce{2H}, \ce{CH2_{pip}}), 1.38\,--\,1.18~(м,~\ce{20H}, \ce{2tBu_{TATBS}}, \ce{CH2_{pip}}), 1.05~(c,~\ce{6H}, \ce{2CH3_{isoph}}).
    \data{ЯМР~\ce{^19F}}[ацетон-d\textsubscript{6}]~\chemdelta,~\si{\ppm}: 18.29, 12.74~(1:1).
    \data{МС:~} Найдено \ce{[M + H]+}: \num{1140.4823}. \ce{C69H69F4N5O2S2}. Рассчитано: \ce{[M + H]} \num{1140.4902}.
\end{experimental}

\textbf{\iupac{\E-1-\{4-[1-(4-\{2-[3-(Дицианометилен)-5,5-диметилциклогекс-1-ен-1-ил]винил\}фенил)-3-фенил-4,5-дигидро-1\H-пиразол-5-ил]-2,3,5,6-тетрафторфенил\}пиперидин-4-ил-3,5-бис(\{[4-(\textit{трет}-бутил)фенил]тио\}метил)-2,4,6-триметилбензоат}~(\cmpd{pentafluoropyrazoline_piperidine_DCIF.IDATBS}).} \todo{пиперидин с одним IDATBS R32}
\textbf{Способ 1.}
Получен по общей методике из \SI{0.09}{\gram}~(\SI{0.14}{\milli\mole}) соединения~\cmpd{pentafluoropyrazoline_DCIF.piperidine}, \SI{0.11}{\gram}~(\SI{0.21}{\milli\mole}) \ce{IDATBS-Cl}, \SI{0.06}{\milli\litre}~(\SI{0.4}{\milli\mole}) триэтиламина и \SI{1}{\milli\gram}~\ac{dmap}.
Время реакции 12 часов.
Темно-красный порошок, выход~\SI{0.01}{\gram}~(\SI{7.5}{\percent}).

\textbf{Способ 2.}
Получен по общей методике c заменой растворителя на ацетонитрил из \SI{0.09}{\gram}~(\SI{0.14}{\milli\mole}) соединения~\cmpd{pentafluoropyrazoline_DCIF.piperidine}, \SI{0.11}{\gram}~(\SI{0.21}{\milli\mole}) \ce{IDATBS-Cl}, \SI{0.06}{\milli\litre}~(\SI{0.4}{\milli\mole}) триэтиламина и \SI{1}{\milli\gram}~\ac{dmap}.
Время реакции 36 часов.
Темно-красный порошок, выход~\SI{0.01}{\gram}~(\SI{7.5}{\percent}).

\textbf{Способ 3.}
Раствор \SI{0.16}{\gram}~(\SI{0.2}{\milli\mole}) соединения~\cmpd{pentafluoropyrazoline_DCIF.piperidine}, \SI{0.20}{\gram}~(\SI{0.4}{\milli\mole}) \ce{IDATBS-Cl}, \SI{0.05}{\milli\litre}~(\SI{0.4}{\milli\mole}) триэтиламина и \SI{1}{\milli\gram}~\ac{dmap} в \SI{6}{\ml} сухого толуола выдерживали при температуре~\SI{130}{\celsius} в микроволновом реакторе.
Растворитель удаляли в вакууме, твердый остаток очищали колоночной хроматографией на \ce{SiO2}, элюент~--- бензол.
Собирали красные фракции, элюент удаляли в вакууме.
Темно-красный порошок, выход~\SI{7}{\milli\gram}~(\SI{2.5}{\percent}).
\begin{experimental}
    \data*{Т\textsubscript*{пл.}} 147--\SI{150}{\celsius}.
    \data{ЭСП}[ацетон] \chemlambda\textsubscript{max}~($\lg \varepsilon$): \SI{497}{\nano\metre}~(4.64).
    \data{ЯМР~\ce{^1H}}[ацетон-d\textsubscript{6}]~\chemdelta,~\si{\ppm}: 7.85\,--\,7.75~(м,~\ce{2H_{Ar}}), 7.60\,--\,7.51~(м,~\ce{2H_{Ar}}), 7.49\,--\,7.38~(м,~\ce{5H_{Ar}}), 7.36\,--\,7.30~(м,~\ce{8H_{IDATBS}}), 7.19~(д,~\ce{1H}, \ce{CH=},~\textit{J}\,=\,16.2~\si{\hertz}), 7.07~(д,~\ce{1H}, \ce{CH=},~\textit{J}\,=\,16.2~\si{\hertz}), 6.75~(с,~\ce{1H}, \ce{=CH_{isoph}}), 5.93~(дд,~\ce{1H_{pyr}},~\textit{J}\,=\,13.0, 5.8~\si{\hertz}), 5.31\,--\,5.16~(м,~\ce{1H}, \ce{CH_{pip}}), 4.20~(с,~\ce{4H}, \ce{2CH2_{IDATBS}}), 4.07\,--\,3.98~(м,~\ce{1H_{pyr}}), 3.52\,--\,3.37~(м,~\ce{3H}, \ce{1H_{pyr}}, \ce{CH2_{pip}}), 3.35\,--\,3.21 ~(м,~\ce{2H}, \ce{CH2_{pip}}), 2.66~(с,~\ce{2H}, \ce{CH2_{isoph}}), 2.60~(с,~\ce{2H}, \ce{CH2_{isoph}}), 2.44~(с,~\ce{3H}, \ce{CH3_{IDATBS}}), 2.30~(с,~\ce{6H}, \ce{2CH3_{IDATBS}}), 2.19\,--\,2.10~(м,~\ce{2H}, \ce{CH2_{pip}}), 1.94\,--\,1.79~(м,~\ce{2H}, \ce{CH2_{pip}}), 1.28~(с,~\ce{18H}, \ce{2tBu_{IDATBS}}), 1.05~(с,~\ce{6H}, \ce{2CH3_{isoph}}).
    \data{ЯМР~\ce{^19F}}[ацетон-d\textsubscript{6}]~\chemdelta,~\si{\ppm}: 18.38, 12.69~(1:1).
    \data{МС:~} Найдено \ce{[M + H]+}: \num{1168.5130}. \ce{C71H73F4N5O2S2}. Рассчитано: \ce{[M + H]} \num{1168.5215}.
\end{experimental}
