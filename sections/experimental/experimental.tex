\section{Экспериментальная часть}

Спектральные данные получены в Исследовательском химическом центре коллективного пользования СО~РАН. Спектры ЯМР регистрировали на спектрометрах Bruker~AV-300 (1H, \SI{300.13}{\mega\hertz}; 19F, \SI{282.37}{\mega\hertz}) и Bruker~AV-400 (1H, \SI{400.13}{\mega\hertz}) в дейтерохлороформе, ДМСО-d6 и ацетоне-d6. Значения химических сдвигов протонов приведены относительно сигналов остаточных протонов растворителей. При регистрации спектров ЯМР~19F в качестве внутреннего стандарта использовали C6F6. Электронные спектры поглощения регистрировали на спектрофотометре Hewlett~Packard~8453, спектры флуоресценции --- на спектрофлуориметре Cary~Eclipсe~(Varian). Масс-спектры высокого разрешения получены на приборе DFS~(Thermo~Fisher~Scientifiс) в режиме прямого ввода, энергия ионизации \SI{70}{\electronvolt}.

\textbf{\IUPAC{4\-(3,5\-бис(2,3,5,6\-тетрафтор-4\-(4\-гидроксипиперидин\-1\-ил)фенил\-4,5-дигидро\-1H-пиразол\-1\-ил)бензальдегид}~(\cmpd{decafluoro.piperidine})}

\textbf{\IUPAC{((1-(4-формилфенил)-4,5-дигидро-1H-пиразол-3,5-диил)бис(2,3,5,6-тетрафтор-4,1-фенилен))бис(пиперидин-1,4-диил)дибензоат}~(\cmpd{benzoate.piperidine})}

\textbf{\IUPAC{4-(3,5-бис(2,3,5,6-тетрафтор-4-(пиперазин-1-ил)фенил)-4,5-дигидро-1H-пиразол-1-ил)бензальдегид}~(\cmpd{decafluoro.piperazine})}

\textbf{\IUPAC{(\textit{E})-((1-(4-(2-(3-(дицианометилен)-5,5-диметилциклогекс-1-ен-1-ил)винил)фенил)-4,5-дигидро-1H-пиразол-3,5-диил)бис(2,3,5,6-тетрафтор-4,1-фенилен))бис(пиперидин-1,4-диил)дибензоат}~(\cmpd{dcif.piperidine})}

\textbf{(\textit{E})-2-(3-(4-(3,5-бис(2,3,5,6-тетрафтор-4-(4-гидроксипиперидин-1-ил)фенил)-4,5-дигидро-1H-пиразол-1-ил)стирил)-5,5-диметилциклогекс-2-ен-1-илиден)малононитрил~(\cmpd{dcif.piperidine})}