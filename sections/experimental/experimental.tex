\section{Экспериментальная часть}

Спектральные данные получены в Исследовательском химическом центре коллективного пользования СО~РАН. 
Спектры ЯМР регистрировали на спектрометрах Bruker~AV-300~(\ce{^1H}, \SI{300.13}{\mega\hertz}; \ce{^19F},~\SI{282.37}{\mega\hertz}) и Bruker~AV-400~(\ce{^1H},~\SI{400.13}{\mega\hertz}) в дейтерохлороформе, ДМСО-\ce{d6} и ацетоне-\ce{d6}. 
Значения химических сдвигов протонов приведены относительно сигналов остаточных протонов растворителей. 
При регистрации спектров ЯМР~\ce{^19F} в качестве внутреннего стандарта использовали~\ce{C6F6}~(\chemdelta{}\textsubscript{F} = \SI{0}{\ppm}). 
Спектры ЯМР~\ce{^13C} регистрировали в режиме широкополосной развязки~(broadband decoupling, BB). 
Электронные спектры поглощения регистрировали на спектрофотометре Hewlett~Packard~8453.
% , спектры флуоресценции --- на спектрофлуориметре Cary~Eclipse~(Varian). 
Масс-спектры высокого разрешения получены на приборе DFS~(Thermo~Fisher~Scientifiс) в режиме прямого ввода, энергия ионизации \SI{70}{\electronvolt}.

В работе использовались следующие коммерчески доступные реактивы и растворители: \todo{написать}

% Хромофор~\cmpd{benzaldehyde_thiophene_TCP.dibuthyl} был получен по методике описанной в \cite{Jang2006}. Соединение идентично описанному по данным спектров ЯМР.
Альдегид~\cmpd{decafluoropyrazoline_CHO} синтезировали по~\cite{2016a,2010}.

% \textbf{\iupac{2-(1-(4-Метил-3,5-бис\{[(2,3,5,5-тетрафтор-4-трифторметилфенил)тио]метил\}-бензоил)-4-\{\E-2-[5-(\E-4-дибутиламиностирил)тиофен-2-ил]винил\}-1,5-дигидро-5-оксо-3-циано-2\H-пиррол-2-илиден)малононитрил}~(\cmpd{benzaldehyde_thiophene_TAFS.dibuthyl}).} \todo{R8} 
% К раствору~\SI{0.10}{\gram}~(\SI{0.20}{\mmol}) соединения~\cmpd{benzaldehyde_thiophene_TCP.dibuthyl} в~\SI{6}{\milli\litre} ацетонитрила в атмосфере аргона прибавляли~\SI{0.08}{\gram}~(\SI{0.8}{\mmol}) прокаленного карбоната калия и кипятили 10 минут. 
% После этого добавляли~\SI{0.14}{\gram}~(0.21 ммоль) ТAFSCl и кипятили 30 минут. 
% Далее добавили еще столько же TAFSCl и кипятили еще 30 минут. Ацетонитрил удаляли в вакууме, твердый остаток очищали колоночной хроматографией на силикагеле, элюент~—-- \ce{CH2Cl2}. 
% Собирали сине-зеленую фракцию, элюент удаляли в вакууме, остаток промывали диэтиловым эфиром. 
% Темно-синий порошок, выход 0.05 г~(\SI{22}{\percent}).
% \begin{experimental}
%     \data{Т\textsubscript{пл.}} $\approx$ \SI{220}{\celsius} (с разложением).
%     \data{ЭСП}[\ce{CHCl3}] \chemlambda\textsubscript{max}~\SI{936}{\nano\metre}.
%     \data{МС}[MALDI-TOF] Найдено \ce{[M+ - H]}: \num{1148.33}. \ce{C54H37F14N5O2S3}. Рассчитано: \ce{M} \num{1149.19}.
%     \data{ЯМР~\ce{^1H}}[\ce{CDCl3}]~$\delta$,~\si{\ppm}: 8.46~(д,~\ce{1H},~\ce{CH=},~\textit{J}~=~14.8~\si{\hertz}), 7.54~(с,~\ce{2H},~\ce{2H_{Ar}}), 7.43~(д,~\ce{1H},~\ce{H_{thio}},~\textit{J}~=~4.2~\si{\hertz}), 7.40~(д,~\ce{2H},~\ce{2H_{Ar}},~\textit{J}~=~8.9~\si{\hertz}), 7.19~(д,~\ce{1H},~\ce{CH=},~\textit{J}~=~15.7~\si{\hertz}), 7.08~(д,~\ce{1H},~\ce{H_{thio}},~\textit{J}~=~4.2~\si{\hertz}), 7.01~(д,~\ce{1H},~\ce{CH=},~\textit{J}~=~15.7~\si{\hertz}), 6.83~(д,~\ce{1H},~\ce{CH=},~\textit{J}~=~14.8~\si{\hertz}), 6.62~(д,~\ce{2H},~\ce{2H_{Ar}},~\textit{J}~=~8.9~\si{\hertz}), 4.27~(с,~\ce{4H},~\ce{2Ar-CH2}), 3.38~--~3.27~(м,~\ce{4H},~\ce{2N-CH2}), 2.68~(с,~\ce{3H},~\ce{Ar-CH3}), 1.67~--~1.53~(м,~\ce{4H},~\ce{2CH2}), 1.45~--~1.26~(м,~\ce{4H}, \ce{2CH2}), 1.01~--~0.90~(м,~\ce{6H},~\ce{2CH3}).
%     \data{ЯМР~\ce{^19F}}[\ce{CDCl3}]~$\delta$,~\si{\ppm}: ]105.57~--~105.31~(м,~\ce{6F}), 30.41~--~29.99~(м,~\ce{4F}), 22.47~--~22.05~(м,~\ce{4F}).
% \end{experimental}

% \textbf{\iupac{\E-1-{4-[2-(Тиофен-2-ил)винил]фенил}пиперидин}~(\cmpd{benzaldehyde_thiophene.piperidine})\todo{R9}.}~
% К суспензии \SI{2.79}{\gram}~(\SI{6.34}{\mmol}) \iupac{бромида трифенил(тиофен-2-илметил)фосфония} в \SI{8}{\ml} сухого ТГФ прибавляли в атмосфере аргона \SI{1.00}{\gram}~(\SI{5.28}{\mmol}) альдегида~\cmpd{benzaldehyde.piperidine}, \SI{2.10}{\gram}~(52.8 ммоль) гидрида натрия~(\SI{60}{\percent} суспензия в минеральном масле) и еще 8 мл ТГФ. 
% Смесь перемешивали 36 часов при комнатной температуре, выливали в 300 мл насыщенного водного раствора хлорида аммония со льдом, осадок отфильтровывали, промывали водой до нейтральной реакции и гексаном. 
% Высушенный осадок очищали колоночной хроматографией на силикагеле, элюент — хлористый метилен : гексан, градиент 1:3 —- 1:0. 
% Собирали желтые флуоресцирующие фракции. 
% Повторно очищали колоночной хроматографией на силикагеле, элюент — хлористый метилен : гексан, градиент 1:2 —- 2:1. 
% Темно-желтый порошок, выход \SI{0.82}{\gram}~(\SI{58}{\percent}), 
% \begin{experimental}
%     \data{Т\textsubscript{пл.}} 147--\SI{150}{\celsius}.
%     \data{ЯМР~\ce{^1H}}[\ce{CDCl3}]~$\delta$,~\si{\ppm}: 7.34~(д,~\ce{2H},~\ce{2H_{Ar}},~\textit{J}~=~8.6~\si{\hertz}), 7.11~(д,~\ce{1H},~\ce{H_{thio}},~\textit{J}~=~4.7~\si{\hertz}), 7.06~(д,~\ce{1H},~\ce{CH=},~\textit{J}~=~16.1~\si{\hertz}), 7.00~--~6.80~(м,~\ce{5H},~\ce{2H_{Ar}, 2H_{thio}, \ce{CH=}}), 3.25~--~3.09~(м,~\ce{4H},~\ce{2CH2}), 1.69~(с,~\ce{6H},~\ce{3CH2}).
% \end{experimental}

% \textbf{\iupac{\E-5-[4-(Пиперидин-1-ил)стирил]тиофен-2-альдегид}~(\cmpd{benzaldehyde_thiophene_CHO.piperidine}).}\todo{R11}~
% Раствор \SI{0.70}{\gram}~(\SI{2.6}{\mmol}) соединения~\cmpd{benzaldehyde_thiophene.piperidine} в \SI{15}{\ml} сухого ТГФ в атмосфере аргона охлаждали до \SI{-80}{\celsius} и прибавляли по каплям \SI{1.56}{\ml}~(\SI{3.9}{\mmol}) н-бутиллития~(\SI{2.5}{\Molar} раствор в гексане) в \SI{4}{\ml} ТГФ, выдерживали при этой температуре 1 час. 
% Прибавляли по каплям раствор \SI{0.3}{\ml}~(\SI{3.9}{\mmol}) ДМФА в \SI{4}{\ml} ТГФ и выдерживали еще час при той же температуре. 
% Затем убирали баню и оставляли нагреваться до комнатной температуры, прибавляли по каплям \SI{10}{\ml} воды и оставляли на ночь. 
% На следующий день отгоняли ТГФ, осадок из водной фазы отфильтровывали, фильтрат экстрагировали 2 раза по \SI{40}{\ml} хлористого метилена, сушили \ce{MgSO4}, растворитель удаляли в вакууме. Осадки объединяли и очищали колоночной хроматографией на силикагеле, элюент~—-- хлористый метилен. 
% Собирали оранжевую фракцию. Темно-оранжевый порошок, выход \SI{0.65}{\gram}~(\SI{84}{\percent}).
% \begin{experimental}
%     \data{Т\textsubscript{пл.}} 169--\SI{171}{\celsius}.
%     \data{ЯМР~\ce{^1H}}[\ce{CDCl3}]~$\delta$,~\si{\ppm}: 9.80~(с,~\ce{1H},~\ce{CHO}), 7.61~(д,~\ce{1H},~\ce{H_{thio}},~\textit{J}~=~3.9~\si{\hertz}), 7.37~(д,~\ce{2H},~\ce{2H_{Ar}},~\textit{J}~=~8.8~\si{\hertz}), 7.11~--~6.98~(м,~\ce{3H},~\ce{CH=CH}, система \emph{AB}, \ce{H_{thio}}), 6.87~(д,~\ce{2H},~\ce{2H_{Ar}},~\textit{J}~=~8.8~\si{\hertz}), 3.31~--~3.15~(м,~\ce{6H},~\ce{3CH2}), 1.73~--~1.64~(м,~\ce{4H},~\ce{2CH2}).
% \end{experimental}

% \textbf{\iupac{2-[4-(\E-2-\{5-[\E-4-(Пиперидин-1-ил)стирил]тиофен-2-ил\}винил)-1,5-дигидро-5-оксо-3-циано-2\H-пиррол-2-илиден]малононитрил}~(\cmpd{benzaldehyde_thiophene_TCP.piperidine}).} \todo{R12}~
% К раствору 0.33 г (2.5 ммоль) димера малононитрила в 5 мл абсолютного этанола в атмосфере аргона прибавляли 0.47 мл (4.2 ммоль) этилпирувата. 
% Реакционную смесь кипятили 1 час и прибавляли 0.50 г соединения~\cmpd{benzaldehyde_thiophene_CHO.piperidine} в 10 мл абсолютного этанола. 
% Кипятили еще 1 час, после этого этанол отгоняли в вакууме, остаток промывали хлористым метиленом. 
% Темно-синий порошок, выход 0.33 г (\SI{43}{\percent}), использовали дальше без дополнительной очистки.
% \begin{experimental}
%     \data{ЭСП}[\ce{CHCl3}] \chemlambda\textsubscript{max}~\SI{838}{\nano\metre}.
% \end{experimental}

\textbf{\iupac{4-{3,5-Бис[2,3,5,6-тетрафтор-4-(4-гидроксипиперидин-1-ил)фенил]-4,5-дигидро-1\H-пиразол-1-ил}бензальдегид}~(\cmpd{decafluoropyrazoline_substituted.piperidine}).}\todo{R19} 
Раствор \SI{3.00}{\gram}~(\SI{5.9}{\mmol}) альдегида~\cmpd{decafluoropyrazoline_CHO} и \SI{1.80}{\gram}~(\SI{17.8}{\mmol}) 4-гидроксипиперидина в \SI{50}{\milli\litre} сухого \ac{dmf} нагревали до \SI{100}{\celsius}, выдерживали при этой температуре 6 часов и оставляли на ночь. 
Реакционную смесь выливали в \SI{400}{\milli\litre} воды со льдом, перемешивали до таяния льда и отфильтровывали осадок. 
Осадок на фильтре промывали водой до нейтральной реакции, затем гексаном и сушили на воздухе. 
Желто-оранжевый порошок, выход \SI{3.70}{\gram}~(\SI{93}{\percent}). 
Продукт очищали колоночной хроматографией на \ce{SiO2}, элюент~--- \ce{CH2Cl2}\,:\,ацетонитрил, градиент 5:1~-- 2:3. Собирали желтые фракции, анализировали ТСХ (\ce{CH2Cl2}\,:\,ацетонитрил, 2:1, R\textsubscript{f} $\approx$ 0.25 -- 0.3).
\begin{experimental}[format=\bfseries]
    \data*{Т\textsubscript*{пл.}} 155--\SI{159}{\celsius}.
    \data{МС}[DFS] Найдено \ce{[M+]}: \num{668.2023}. \ce{C32H28O3N4F8}. Рассчитано: \ce{M} \num{668.2028}.
    \data{ЯМР~\ce{^1H}}[ацетон-\ce{d6}] $\delta$, \si{\ppm}: 9.77~(с,~\ce{1H},~\ce{CHO}), 7.76~(д,~\ce{2H},~\ce{2H_{Ar}},~\textit{J}~=~8.8~\si{\hertz}), 7.17~(д,~\ce{2H},~\ce{2H_{Ar}},~\textit{J}~=~8.8~\si{\hertz}), 5.98,~4.16,~3.90~(все дд, все по~\ce{1H}, система~\emph{ABX} пиразолина,~\textit{J}~=~18.2, 13.1, 5.3~\si{\hertz}), 3.86~--~3.69~(м,~\ce{3H}), 3.63~--~3.47~(м,~\ce{2H}), 3.47~--~3.31~(м,~\ce{2H}), 3.28~--~3.17~(м,~\ce{2H}), 3.17~--~3.03~(м,~\ce{2H}), 2.00~--~1.82~(м,~\ce{4H}), 1.77~--~1.50~(м,~\ce{4H}).
    \data{ЯМР~\ce{^19F}}[ацетон-\ce{d6}] $\delta$, м.д.: 22.27~(дд,~\ce{2F},~\textit{J}~=~18.4, 6.9~\si{\hertz}), 18.05~(уш.~c.,~\ce{2F}), 12.62~(дд,~\ce{2F},~\textit{J}~=~20.8, 6.5~\si{\hertz}), 11.67~(дд,~\ce{2F},~\textit{J}~=~17.6, 6.0~\si{\hertz}).
    \data{ЯМР~\ce{^13C}}[\ce{CDCl3}] $\delta$, м.д.: 190.41, 147.40, 146.52, 146.02, 144.49, 144.05, 142.98, 141.03, 140.61, 131.67, 131.03, 130.93, 128.59, 112.66, 110.35, 110.22, 110.10, 104.00, 77.15, 76.90, 76.64, 67.18, 67.15, 51.57, 48.54, 48.51, 48.48, 48.44, 43.57, 34.85, 34.83.
\end{experimental}


\textbf{{[1-(4-Формилфенил)-4,5-дигидро-1H-пиразол-3,5-диил]бис(2,3,5,6-тетрафтор-4,1-фенилен)}бис(пиперидин-1,4-диил)дибензоат~(\cmpd{decafluoropyrazoline_benzoyl.piperidine}).}\todo{R20} \textbf{Способ 1} Суспензию \SI{0.50}{\gram}~(\SI{0.75}{\mmol}) альдегида~\textbf{\cmpd{decafluoropyrazoline_substituted.piperidine}} в \SI{10}{\milli\litre} сухого бензола доводили до кипения и прибавляли к ней \SI{0.62}{\milli\litre}~(\SI{4.5}{\mmol}) триэтиламина и \SI{0.35}{\milli\litre}~(\SI{3.0}{\mmol}) хлористого бензоила. После двух часов кипячения прибавляли еще столько же триэтиламина и хлористого бензоила и кипятили еще сутки. Реакционную смесь выливали в \SI{100}{\milli\litre} воды и добавляли бензол до разделения фаз. Органическую фазу отделяли, сушили над \ce{Na2SO4} и удаляли растворитель в вакууме. Твердый остаток очищали колончной хроматографией на \ce{SiO2}, элюент~--- бензол\,:\,\ce{CHCl3}, градиент 1:0~--~0:1. Собирали желтые фракции, элюент удаляли в вакууме и повторно очищали колончной хроматографией на \ce{SiO2}, элюент~--- смесь бензол\,:\,\ce{CH2Cl2}~1:1. Собирали желтые фракции, растворитель удаляли в вакууме. Желтое масло, выход \SI{0.49}{\gram}~(\SI{74}{\percent}).

\textbf{Способ 2} К суспензии \SI{0.20}{\gram}~(\SI{0.3}{\mmol}) альдегида~\textbf{\cmpd{decafluoropyrazoline_substituted.piperidine}} в \SI{5}{\milli\litre} сухого бензола, прибавляли~\SI{0.11}{\milli\litre}~(\SI{0.75}{\mmol}) хлористого бензоила, \SI{0.13}{\milli\litre}~(\SI{0.75}{\mmol}) триэтиламина и \SI{2}{\milli\gram}~\ac{dmap}. 
Реакционную смесь кипятили 6 часов, оставляли на ночь и удаляли растворитель в вакууме. 
Полученное масло очищали колоночной хроматографией на \ce{SiO2}, элюент~--- смесь ацетонитрил\,:\,\ce{CH2Cl2}, градиент 1:1 -- 8:1, собирали желтую фракцию, элюент удаляли в вакууме, полученное масло промывали смесью гексана с диэтиловым эфиром 1:1. Светло-желтый порошок, выход~\SI{0.19}{\gram}~(\SI{74}{\percent}).
\begin{experimental}
    \data*{Т\textsubscript*{пл.}} 180--\SI{183}{\celsius}.
    \data{МС}[DFS] Найдено \ce{[M+]}: \num{876.2548}. \ce{C46H36O5N4F8}. Рассчитано: \ce{M} \num{876.2553}.
    \data{ЯМР~\ce{^1H}}[\ce{CDCl3}]~$\delta$,~м.д.: 9.77~(с,~\ce{1H},~\ce{CHO}), 8.00 -- 8.14 ~(м,~\ce{4H_{Ar}}), 7.73~(д,~\ce{2H_{Ar}},~\textit{J}~=~8.4~\si{\hertz}), 7.61~--~7.52~(м,~\ce{2H_{Ar}}), 7.50~--~7.39~(м,~\ce{4H_{Ar}}), 7.13~(д,~\ce{2H_Ar},,~\textit{J}~=~8.4~\si{\hertz}), 5.75, 3.95~(оба дд,~оба по \ce{1H}, пиразолин,~\textit{J}~=~17.8, 13.0, 5.9~\si{\hertz}), 5.35 -- 5.11~(м,~\ce{3H}, \ce{2CH-OH}, пиразолин), 3.65~--~3.41~(м,~\ce{4H}), 3.41~--~3.13~(м,~\ce{4H}), 2.26~--~2.03~(м,~\ce{4H}), 2.03~--~1.87~(м,~\ce{4H}).
    \data{ЯМР~\ce{^19F}}[CDCl3]~$\delta$,~\si{\ppm}: 21.14~(д,~\ce{2F},~\textit{J}~=~12.2~\si{\hertz}), 16.72~(уш. с,~\ce{2F}), 11.74~(с,~\ce{2F}), 11.14~--~9.71~(м,~\ce{2F}).
\end{experimental}

% \textbf{4-(3,5-бис(2,3,5,6-тетрафтор-4-(пиперазин-1-ил)фенил)-4,5-дигидро-1H-пиразол-1-ил)бензальдегид~(\cmpd{decafluoropyrazoline_substituted.piperazine}).}\todo{R21} В \SI{10}{\milli\litre} сухого \ac{dmf} растворили \SI{0.50}{\gram}~(\SI{1}{\mmol}) соединения~\cmpd{decafluoropyrazoline_CHO} и \SI{0.27}{\gram}~(\SI{3}{\mmol}) пиперазина. 
% Реакционную смесь выдерживали при \SI{100}{\celsius} в течение 4 часов.
% После охлаждения выливали в \SI{100}{\milli\litre} воды, осадок отфильтровывали и сушили на воздухе. \todo{Включать?}

\textbf{\iupac{\E-\{[1-(4-\{2-[3-(Дицианометилен)-5,5-диметилциклогекс-1-ен-1-ил]винил\}фенил)-4,5-дигидро-1\H-пиразол-3,5-диил]бис(2,3,5,6-тетрафтор-4,1-фенилен)\}бис(пиперидин-1,4-диил)дибензоат}~(\cmpd{decafluoropyrazoline_piperidine_DCIF.benzoyl}).}\todo{R22} \textbf{Способ 1.} К суспензии \SI{0.48}{\gram}~(\SI{0.55}{\mmol}) альдегида~\textbf{\cmpd{decafluoropyrazoline_benzoyl.piperidine}} в \SI{15}{\milli\litre} бутанола прибавляли \SI{0.10}{\gram} дицианоизофорона и 5 капель морфолина. Смесь кипятили в атмосфере аргона 7 часов, растворитель удаляли в вакууме. Твердый остататок очищали колончной хроматографией на \ce{SiO2}, элюент~--- \ce{CH2Cl2}\,:\,гексан, градиент 1:1~--~0:1, затем ацетонитрил. Собирали красные фракции.

\textbf{Способ 2}. \todo{R24} К суспензии \SI{0.10}{\gram}~(\SI{0.12}{\mmol}) соединения~\textbf{\cmpd{decafluoropyrazoline_DCIF.piperidine}} в \SI{5}{\milli\litre} сухого бензола прибавляли \SI{0.35}{\milli\litre}~(\SI{0.30}{\mmol}) хлористого бензоила, \SI{0.42}{\milli\litre}~(\SI{0.30}{\mmol}) триэтиламина и \SI{7}{\milli\gram} \ac{dmap}. 
Реакционную смесь кипятили в атмосфере аргона 10 часов, добавив еще столько же хлористого бензоила. 
Растворитель удаляли в вакууме. Очищали колоночной хроматографией на \ce{SiO2}, элюент~--- смесь ацетонитрил\,:\,\ce{CH2Cl2}, градиент 1:10 -- 1:1.
Собирали оранжевые фракции, растворитель удаляли в вакууме. Темно-оранжевый порошок, выход

\textbf{\iupac{\E-2-[3-(4-\{3,5-Бис[2,3,5,6-тетрафтор-4-(4-гидроксипиперидин-1-ил)фенил]-4,5-дигидро-1\H-пиразол-1-ил\}стирил)-5,5-диметилциклогекс-2-ен-1-илиден]малононитрил}~(\cmpd{decafluoropyrazoline_DCIF.piperidine}).}\todo{R23} К раствору \SI{0.25}{\gram}~(\SI{0.37}{\mmol}) альдегида~\textbf{\cmpd{decafluoroaldehyde.piperidine}} и \SI{0.070}{\gram}~(\SI{0.37}{\mmol}) дицианоизофорона в \SI{5}{\milli\litre} бутанола прибавляли 5 капель морфолина, кипятили в атмосфере аргона 7 часов и оставляли на ночь. Выпавший осадок отфильтровывали, промывали этанолом и диэтиловым эфиром. Темно-красный порошок, выход \SI{0.13}{\gram}~(\SI{42}{\percent}).

\textbf{Моноэфиры~(общая методика)}
К раствору \SI{0.10}{\gram} соединений \cmpd{, pentafluoropyrazoline_DCIF.piperidine} в \SI{6}{\milli\litre} сухого бензола добавялили 1.25--1.5~экв. хлорангидридов \todo{дендроиды}, 2-3~экв. триэтиламина и 0.05~экв. \ac{dmap}.
Полученную смесь кипятили до окончания реакции. 
Растворитель удаляли в вакууме, твердый остаток очищали колончной хроматографией на \ce{SiO2}, элюент~--- бензол. 
Элюент удаляли в вакууме, твердый продукт промывали гексаном или смесью гексан-эфир.

\textbf{\iupac{\E-\{[1-(4-\{2-[3-(Дицианометилен)-5,5-диметилциклогекс-1-ен-1-ил]винил\}фенил)-4,5-дигидро-1\H-пиразол-3,5-диил]бис(2,3,5,6-тетрафтор-4,1-фенилен)\}бис(пиперидин-1,4-диил) бис[4-метил-3,5-бис(\{[2,3,5,6-тетрафтор-4-(трифторметил)фенил]тио\}метил)бензоат]}~(\cmpd{decafluoropyrazoline_piperidine_DCIF.TAFS}).} \todo{пиперидин с двумя TAFS R30} 
По общей методике из \SI{0.10}{\gram}~(\SI{0.12}{\milli\mole}), \SI{0.24}{\gram}~(\SI{0.36}{\milli\mole}, 3~экв.) TAFSCl, \SI{0.10}{\milli\litre}~(\SI{0.72}{\milli\mole}, 6~экв.) триэтиламина и \SI{0.001}{\gram}~(0.05~экв.) \ac{dmap}. 
Время реакции 2 часа. Твердый остаток после удаления растворителя очищали колоночной хроматографией на \ce{SiO2}, элюент~--- \ce{CH2Cl2}.
Темно-красный порошок, выход \SI{0.075}{\gram}\todo{посмотреть после очистки}~(\SI{30}{\percent}).

\textbf{\iupac{\E-\{[1-(4-\{2-[3-(Дицианометилен)-5,5-диметилциклогекс-1-ен-1-ил]винил\}фенил)-4,5-дигидро-1\H-пиразол-3,5-диил]бис(2,3,5,6-тетрафтор-4,1-фенилен)\}бис(пиперидин-1,4-диил) бис[3,5-бис(\{[4-(трет-бутил)фенил]тио\}метил)-4-метилбензоат]}~(\cmpd{decafluoropyrazoline_piperidine_DCIF.TATBS}).}\todo{пиперидин с двумя TATBS R33} 
По общей методике из \SI{0.06}{\gram}~(\SI{0.07}{\milli\mole}), \SI{0.12}{\gram}~(\SI{0.22}{\milli\mole}) TATBSCl, \SI{0.14}{\milli\litre}~(\SI{1.1}{\milli\mole}) триэтиламина и \SI{0.001}{\gram}~(0.05~экв.) \ac{dmap}. 
Время реакции 16 часов. Твердый остаток после удаления растворителя очищали колоночной хроматографией на \ce{SiO2}, элюент~--- бензол.
Темно-красный порошок, выход \SI{0.070}{\gram}~(\SI{55}{\percent}).

\textbf{\iupac{\E-1-\{4-[1-(4-\{2-[3-(Дицианометилен)-5,5-диметилциклогекс-1-ен-1-ил]винил\}фенил)-3-фенил-4,5-дигидро-1\H-пиразол-5-ил]-2,3,5,6-тетрафторфенил\}пиперидин-4-ил 3,5-бис(\{[4-(трет-бутил)фенил]тио\}метил)бензоат}~(\cmpd{pentafluoropyrazoline_piperidine_DCIF.TATBS}).} \todo{пиеридин с одним TATBS R31}
По общей методике из \SI{0.10}{\gram}~(\SI{0.15}{\milli\mole}), \SI{0.12}{\gram}~(\SI{0.23}{\milli\mole}) TATBSCl, \SI{0.06}{\milli\litre}~(\SI{0.4}{\milli\mole}) триэтиламина и \SI{0.001}{\gram}~(0.05~экв.) \ac{dmap}. 
Время реакции 3 часов. Твердый остаток после удаления растворителя очищали колоночной хроматографией на \ce{SiO2}, элюент~--- бензол.
Темно-красный порошок, выход \SI{0.10}{\gram}~(\SI{59}{\percent}).

\textbf{\iupac{\E-1-\{4-[1-(4-\{2-[3-(Дицианометилен)-5,5-диметилциклогекс-1-ен-1-ил]винил\}фенил)-3-фенил-4,5-дигидро-1\H-пиразол-5-ил]-2,3,5,6-тетрафторфенил\}пиперидин-4-ил 3,5-бис(\{[4-(трет-бутил)фенил]тио\}метил)-2,4,6-триметилбензоат}~(\cmpd{pentafluoropyrazoline_piperidine_DCIF.MATBS}).} \todo{пиеридин с одним MATBS R32}
По общей методике из \SI{0.09}{\gram}~(\SI{0.14}{\milli\mole}), \SI{0.11}{\gram}~(\SI{0.21}{\milli\mole}) MATBSCl, \SI{0.06}{\milli\litre}~(\SI{0.4}{\milli\mole}) триэтиламина и \SI{0.001}{\gram}~(0.05~экв.) \ac{dmap}. 
Время реакции 12 часов. Твердый остаток после удаления растворителя очищали колоночной хроматографией на \ce{SiO2}, элюент~--- бензол.
Темно-красный порошок, выход \SI{0.012}{\gram}~(\SI{7.5}{\percent}).