\section{Введение}
Увеличивающиеся объемы передаваемой информации ставят задачу создания новых методов ее обработки, в том числе оптических. Большую перспективу имеют электрооптические~(ЭО) модуляторы~--- устройства сопрягающие оптические и электрические линии передачи сигналов. Рабочей средой в таких устройствах является органический донорно-акцепторный хромофор, проявляющий нелинейность второго порядка, заключенный в полимерную матрицу.

Отличительным свойством органических нелинейно-оптических~(НЛО) материалов является возможность получения хромофоров, поглощающих в заданной области спектра. В настоящее время актуальны разработки материалов, работающих в широкой спектральной области~–-- от 500 нм до 1550 нм.

Органические~НЛО материалы обладают важным преимуществом относительно неорганических~--- на порядок более высокими значениями~НЛО восприимчивости второго порядка и, соответственно, меньшими величинами управляющих напряжений, и относительно полупроводниковых~—-- высокой температурной стабильностью спектральных~ЭО свойств. Также они отличаются сверхбыстрым откликом, возможностью изменять их свойства в широких пределах и формировать из них планарные микроструктуры.

В контексте НЛО свойств также важны взаимодействия между хромофором и полимерной матрицей, так как эффективность, температурная и временная стабильность  ЭО модулятора зависит, в том числе, от эффективности ориентации молекул хромофора в полимерной матрице. С этой целью в структуру хромофоров вводятся разветвленные~(дендроидные) заместители, препятствующие агрегации молекул хромофора в полимере при больших концентрациях. \todo{криво}

Для поиска новых~НЛО сред, исследования взаимодействий хромофора с матрицей необходимы хромофоры с максимум поглощения около \SI{532}{\nano\metre}~(излучение Nd:YAG лазера).

Синтез хромофоров для ЭО модуляторов является одним из основных направлений научной тематики в Лаборатории органических светочувствительных материалов НИОХ СО РАН. В качестве таких хромофоров используются полиметиновые красители биполярной структуры с различной длиной полиметиновой цепи. Ранее в лаборатории был синтезирован ряд новых хромофоров для спектральной области 720\,--\,760 и 500\,--\,550 \si{\nano\metre} с использованием полифторированных триарилпиразолинов в качестве донорных блоков~\cite{2019, 2019b}.

\textbf{Цель и задачи работы.}
Исходя из описанного была сформулирована цель данной работы~--- синтез новых нелинейных хромофоров на основе полифторированных триарилпиразолинов.
Были поставлены следующие задачи:
\begin{enumerate}
    \item Разработать подход к синтезу нелинейных хромофоров с разветвленными заместителями на основе полифторированных триарилпиразолинов, замещенных бифункциональными нуклеофилами.
    \item Оптимизировать методику введения в молекулу хромофора разветвленных заместителей.
    \item Наработать нелинейные хромофоры на основе полифторированных триарилпиразолинов, замещенных бифункциональными нуклеофилами, в количествах, достаточных для дальнейших исследований.
\end{enumerate}

\textbf{Вклад автора.}
Дипломная работа полностью выполнена автором.
Подбор и анализ литературы по теме «Синтез и свойства пиразолинов», написание обзора сделаны автором полностью самостоятельно.
Планирование и проведение эксперимента, обсуждение и анализ полученных результатов требовали минимального участия руководителя.
Экспериментальная часть выполнена полностью автором.
Всего в работе получено, выделено и охарактеризовано 11 новых соединений.