\section{Введение}
Увеличивающиеся объемы передаваемой информации ставят задачу создания новых методов ее обработки, в том числе оптических. Большую перспективу имеют электрооптические~(ЭО) модуляторы, основанные на композициях хромофор-полимер. Рабочей средой в таких устройствах является органический донорно-акцепторный хромофор, проявляющий нелинейность второго порядка.

Органические нелинейно-оптические~(НЛО) материалы обладают важным преимуществом относительно неорганических — бóльшими значениями НЛО восприимчивости второго порядка и, соответственно, меньшими величинами управляющих напряжений, и относительно полупроводниковых~—-- высокой температурной стабильностью спектральных~ЭО свойств.

Отличительным свойств органических~НЛО материалов является возможность получения хромофоров, поглощающих в заданной области спектра. В настоящее время актуальны разработки материалов, работающих в двух спектральных областях: 1300--1550~\si{\nano\metre}~(область нулевой дисперсии кварцевого оптического волокна) и 830--900 нм (окно прозрачности атмосферы).

Важными являются также пленкообразующие свойства органических хромофоров, так как эффективность работы ЭО модулятора зависит, в том числе, от эффективности ориентации молекул хромофора в полимерной матрице. С этой целью в структуру хромофоров вводятся разветвленные~(дендроидные) заместители, препятствующие агрегации молекул хромофора в полимере при больших концентрациях.

Синтез хромофоров для ЭО модуляторов является одним из основных направлений научной тематики в Лаборатории органических светочувствительных материалов НИОХ СО РАН. В качестве таких хромофоров используются полиметиновые красители биполярной структуры с различной длиной полиметиновой цепи. Ранее в лаборатории был синтезирован ряд новых хромофоров для спектральной области 720--760 \si{\nano\metre} с использованием полифторированных триарилпиразолинов в качестве донорных блоков~\cite{2019}.

Целью данной работы является синтез новых нелинейных хромофоров на основе полифторированных триарилпиразолинов. Таким образом, были сформулированы следующие задачи:
\begin{enumerate}
    \item Разработать подход к синтезу нелинейных хромофоров на основе полифторированных триарилпиразолинов, замещенных бифункциональными нуклеофилами для области 500--600\si{\nano\metre} и для ИК-области.
    \item Оптимизировать методику введения в молекулу хромофора дендроидных заместителей.
\end{enumerate}

\textbf{Вклад автора.}
% Автором лично были получены все соединения, за исключением отдельно указанных. 
\todo{Что написать?}