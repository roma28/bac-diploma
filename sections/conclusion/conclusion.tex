\section{Выводы}
% В ходе работы были достигнуты следующие результаты:
\begin{enumerate}
    \item Исследована реакция \iupac{4­-[3,5-­бис(пентафторфенил)-­4,5­-дигидро-­1\H-­пиразол­-1-­ил]бензальдегида} с \mbox{4-гидроксипиперидином} в диметилформамиде. Установлено, что пентафторфенильное кольцо в положении 5 пиразолинового цикла более реакционноспособно по отношению к нуклеофилу, чем \ce{C6F5}-группа в положении 3.
    \item Изучены две последовательности стадий в синтезе красителей на примере реакции бензоилирования; показано, что  ацилирование гидроксипиперидиногруппы в красителе предпочтительнее, чем в исходном альдегиде.
    \item Синтезирован ряд новых красителей с одним и двумя замещенными полифторированными кольцами конденсацией формильных производных трифенилпиразолинов с дицианоизофороном и их последующим ацилированием.
    \item Испытаны различные реагенты и катализаторы для введения разветвленных фрагментов в структуру полученных красителей. Показано, что ацилирование хлорангидридом соответствующей кислоты успешно протекает в присутствии \mbox{4-диметиламинопиридина}~(\ac{dmap}), а использование кислоты в качестве ацилирующего агента дает высокий выход в условиях реакции Мицунобу.
    \item Наработаны образцы новых красителей, модифицированных разветвленными фрагментами, в количествах, необходимых для дальнейших исследований.
\end{enumerate}



% \item Изучено взаимодействие формилированного декафтортрифенилпиразолина с бифункциональными нуклеофилами~--- \mbox{4-гидроксипиперидином} и пиперазином. Показано, что при нагревании в диметилформамиде с \mbox{4-гидроксипиперидином} образуется продукт замещения \emph{пара}-атомов фтора, при этом реакция происходит по атомам азота. Выделен и охарактеризован побочный продукт реакции. Также показана разница в реакционной способности пентафторфенильных колец в положениях 3 и 5 пиразолина в реакции ароматического нуклеофильного замещения.
%     \item На модели реакции бензоилирования исследовано образование эфиров \mbox{4-гидроксипиперидинового} производного формилированного декафтортрифенилпиразолина и продукта его конденсации с дицианоизофороном. Разработана и оптимизирована методика получения дицианоизофороновых красителей на основе формилированных декафтортрифенилпиразолина и пентафтортрифенилпиразолина с разделительными блоками на основе \emph{пара}-толуиловой и \chembeta-изодуриловой кислот.
%     \item Исследованы альтернативные методы введения разделительного блока~--- реакция Мицунобу и реакция Штеглиха. Показана перспективность подхода, основанного на реакции Мицунобу.