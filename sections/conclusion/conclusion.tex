\section{Заключение}
В ходе работы были достигнуты следующие результаты:
\begin{enumerate}
    \item Изучено взаимодействие формилированного декафтортрифенилпиразолина с бифункциональными нуклеофилами~--- \mbox{4-гидроксипиперидином} и пиперазином. Показано, что при нагревании в диметилформамиде с \mbox{4-гидроксипиперидином} образуется продукт замещения \emph{пара}-атомов фтора, при этом реакция происходит по атомам азота. Выделен и охарактеризован побочный продукт реакции. Также показана разница в реакционной способности пентафторфенильных колец в положениях 3 и 5 пиразолина в реакции ароматического нуклеофильного замещения.
    \item На модели реакции бензоилирования исследовано образование эфиров \mbox{4-гидроксипиперидинового} производного формилированного декафтортрифенилпиразолина и продукта его конденсации с дицианоизофороном. Разработана и оптимизирована методика получения дицианоизофороновых красителей на основе формилированных декафтортрифенилпиразолина и пентафтортрифенилпиразолина с разделительными блоками на основе \emph{пара}-толуиловой и \chembeta-изодуриловой кислот.
    \item Исследованы альтернативные методы введения разделительного блока~--- реакция Мицунобу и реакция Штеглиха. Показана перспективность подхода, основанного на реакции Мицунобу.
\end{enumerate}

